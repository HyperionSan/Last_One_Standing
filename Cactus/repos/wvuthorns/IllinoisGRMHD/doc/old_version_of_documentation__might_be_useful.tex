\documentclass[showpacs,amsmath,amssymb,prd]{revtex4}
%\documentclass[rmp,aps,tightenlines,12pt,preprint]{revtex4-1}
\usepackage{longtable}
\usepackage{color}
\newcommand{\beq}{\begin{equation}}
\newcommand{\eeq}{\end{equation}}
\newcommand{\beqn}{\begin{eqnarray}}
\newcommand{\eeqn}{\end{eqnarray}}
\newcommand{\half} {{1\over 2}}
\newcommand{\tB}{\tilde{B}}
\newcommand{\tS}{\tilde{S}}
\newcommand{\tF}{\tilde{F}}
\newcommand{\tf}{\tilde{f}}
\newcommand{\tT}{\tilde{T}}
\newcommand{\sg}{\sqrt{\gamma}\,}
\newcommand{\ve}[1]{\mbox{\boldmath $#1$}}

\linespread{1.0}
\usepackage{hyperref}

\newenvironment{packed_itemize}{
\begin{itemize}
  \setlength{\itemsep}{0.0pt}
  \setlength{\parskip}{0.0pt}
  \setlength{\parsep}{ 0.0pt}
}{\end{itemize}}

\newenvironment{packed_enumerate}{
\begin{enumerate}
  \setlength{\itemsep}{0.0pt}
  \setlength{\parskip}{0.0pt}
  \setlength{\parsep}{ 0.0pt}
}{\end{enumerate}}

\bibpunct{(}{)}{;}{a}{,}{,}

\begin{document}
\title{IllinoisGRMHD: A Robust, Open-Source Dynamical Spacetime GRMHD Code
    for Beginners}
\author{Zachariah B. Etienne}
\email{zetienne@illinois.edu}
\affiliation{Physics Department and Joint Space-Science Institute,
University of Maryland, College Park, MD 20742, USA}
\affiliation{NASA Goddard Space Flight Center, 8800 Greenbelt Rd., Greenbelt, MD 20771, USA}

\begin{abstract}
IllinoisGRMHD employs a conservative, high-resolution shock capturing
scheme to evolve the general relativistic magnetohydrodynamics (GRMHD)
equations in a dynamical spacetime context. It is based on the
original code of the Illinois numerical relativity group, but has been
rewritten for the benefit of beginners, with complete documentation
and code comments, as well as a new, highly modular interface. This
code has been released as an open-source module in the latest version
of the Einstein Toolkit.
\end{abstract}

\maketitle

\section{Background}
After about a decade of rapid code development, the Illinois numerical
relativity group's dynamical-spacetime GRMHD code needed work. Though
it was a thoroughly debugged, extremely robust piece of software capable
of doing cutting-edge simulations, 
\begin{enumerate} 
\item It had very few code comments, and only a number of published
  papers as documentation.
\item Many of its original features had not been used
in since the code was first written, leading to many thousands of
lines of effectively ``dead code''.
\item The remaining functioning code was written in a way that was not
  easily extensible, so that, e.g., adding another time-evolved
  variable would require extensive modifications of the code.
\end{enumerate}
At any given time, there were only a
few people at Illinois with sufficient experience to quickly and
effectively extend this closed-source code to do new science, and
given the large amount of dead code and lack of documentation, the
task for beginners to get up to speed was a monumental one, requiring
they decrypt large regions of mostly uncommented code.

Being one of the longest-term core developers, I take the blame
for lack of documentation in the original code. As penance
and for the community's benefit, I decided to rewrite the Illinois
group's GRMHD code and open source it, with the PI's (Stu Shapiro's)
blessing. One goal is to make this code easy for beginners to
understand and modify, with minimal effort. To this end, you will find 
\begin{enumerate}
\item Many code comments, 
\item Easy high-level interfaces to access lower-level functionality, 
\item Low-level functions designed to do a single task (the
\href{http://en.wikipedia.org/wiki/Unix_philosophy}{``Rule of
  Modularity''}), and, of course, 
\item Documentation with detail sufficient to make your eyes roll.
\end{enumerate}
Despite these many nice features, the current version of IllinoisGRMHD
should never be considered perfect. As a starting point, you will find
various comments starting with ``FIXME'', which point to various
janitorial tasks that should be completed. The only guarantee of the
current version of IllinoisGRMHD is that it agrees to roundoff error
with the original version of the GRMHD code from the Illinois group. 

By making IllinoisGRMHD open source (licensed under the GNU GPL v2 or
higher), my hope is that others might find it useful and help
contribute to the open-source version. Even if you do not intend to
modify the code, better documentation and code comments are always
welcome and will benefit everyone!

\section{Basic Equations}
This code is designed to be embedded into a larger code that solves
the coupled Einstein-MHD-Maxwell equations. In particular, this code
solves the MHD-Maxwell equations, or GRMHD eqations, using the
techniques outlined in the next section. We solve these equations
using geometric units of $G=c=1$, and notation such that Latin indices
denote spatial components (1-3) and Greek indices denote spacetime
components (0-3). 

The GRMHD equations in the ideal MHD limit include: the
continuity (Eq.~\ref{eq:Continuity}), energy (Eq.~\ref{eq:Energy}),
momentum (Eq.~\ref{eq:Euler}), and magnetic induction
(Eqs.~\ref{eq:AInduction} and \ref{eq:EMGauge}) equations. The purpose
of this section is to provide a basic roadmap of the GRMHD code, with
minimal discussion motivations and derivations. For detailed
descriptions/derivations of the equations, including definitions of
spacetime metric quantities, please see
\url{http://arxiv.org/abs/astro-ph/0503420} (for
Eqs.~\ref{eq:Continuity}, \ref{eq:Energy}, and \ref{eq:Euler}),
\url{http://arxiv.org/abs/1110.4633} (for Eqs.~\ref{eq:AInduction} and
\ref{eq:EMGauge}).

\begin{center}
\vskip 0.3cm
\fbox{
\parbox{10.5cm}{
\beq
  \partial_t \rho_* = - \partial_j (\rho_* v^j) \ ,
\label{eq:Continuity}
\eeq
\beq
  \partial_t \tilde{\tau} = - \partial_i ( \alpha^2 \sqrt{\gamma}\, T^{0i} 
-\rho_* v^i) + s \ ,
\label{eq:Energy}
\eeq
\beq
  \partial_t \tilde{S}_i = - \partial_j (\alpha \sqrt{\gamma}\,
  T^j{}_i) + \frac{1}{2} \alpha \sqrt{\gamma} \, T^{\alpha \beta} g_{\alpha \beta,i} \ ,
\label{eq:Euler}
\eeq
\beq
\partial_t A_i = \epsilon_{ijk} v^j B^k 
 - \partial_i (\alpha \Phi -\beta^j A_j) \ 
\label{eq:AInduction}
\eeq
\beq
\partial_t (\sqrt{\gamma}\, \Phi) \equiv \partial_t (\psi^6 \Phi)  = 
- \partial_j (\alpha \sqrt{\gamma}\, A^j - \sqrt{\gamma}\, \beta^j \Phi)
- \xi \alpha (\sqrt{\gamma} \Phi)
\label{eq:EMGauge}
\eeq
}}
\end{center}

Here, the energy source term $s$ is given by
\beqn
  s &=& -\alpha \sqrt{\gamma}\, T^{\mu \nu} \nabla_{\nu} n_{\mu}  \cr
   &=& \alpha \sqrt{\gamma}\, [ (T^{00}\beta^i \beta^j + 2 T^{0i} \beta^j 
+ T^{ij}) K_{ij} \cr
 & & - (T^{00} \beta^i + T^{0i}) \partial_i \alpha ]\ .
\label{tau_curvature}
\eeqn


Note that the above time evolution equations are cast into a
``conservative'' form, to conserve rest mass and so that shocks may be
reliably modeled via a high-resolution shock capturing scheme. Also,
to ensure the ``no-monopole'' $\nabla \times \ve{B}=0$ constraint is satisifed, we
choose to rewrite the magnetic induction equation in terms of the the
vector potential $A^{\mu}$, which is defined so that  
\beq 
B^i=\epsilon^{ijk} \partial_j A_k .
\eeq
I.e., the magnetic field is simply the curl of the vector potential
$A_k$. Notice that the magnetic field is unperturbed if we add a term
$\nabla_k (\sqrt{\gamma} \Phi)$ to $A_k$, since the curl of a gradient is zero. Thus
we have the freedom to add such a term to the vector potential. This
is called our ``gauge'' freedom, and although our choice of
$(\sqrt{\gamma} \Phi)$ does not matter in unigrid simulations,
truncation errors from interpolations of $A_k$ on AMR refinement
boundaries will convert purely gauge modes into a combination of gauge
{\it and physical} modes. If the gauge modes do not propagate,
repeated interpolations common on these refinement boundaries will
lead to a build-up of errors, crashing simulations. So we choose to
evolve $(\sqrt{\gamma} \Phi)$ forward in time via the Lorenz gauge
condition (Eq.~\ref{eq:EMGauge}), in which all gauge modes may
propagate, effectively disallowing the build-up of errors on AMR
refinement boundaries. We add an additional damping term to the
standard Lorenz gauge condition, so that modes of $(\sqrt{\gamma}
\Phi)$ damp to zero as $r\to\infty$. This reduces the buildup of
errors on the outer boundary, caused by approximate outer boundary
conditions.

The conservative variables are defined in terms of ``primitive''
variables (such as baryonic density $\rho_b$, pressure $P$, and
four-velocity $u^{\mu}=\{u^0,u^0 v^i\}$) as follows:
\beqn
\rho_* &=& \alpha \sqrt{\gamma} \rho_b u^0 \\
\tilde{S}_i &=&  \alpha \sqrt{\gamma} T^0{}_i \\
\tau &=& \alpha^2 \sqrt{\gamma} T^{00}, \ \text{where} \\
T^{\mu \nu} &=& (\rho_b h +b^2) u^{\mu} u^{\nu} + \left( P +
\frac{b^2}{2} \right) g^{\mu \nu} - b^{\mu} b^{\nu}\ \text{(perfect
  fluid stress-energy tensor)}, \\
b^{\mu} &=& \frac{B^{\mu}}{\sqrt{4\pi}} \ \text{($B^{\mu}$ is defined in
  fluid rest frame)}, \ \text{and}  \\
\tilde{B}^i &=& \sqrt{\gamma} B^i
\eeqn

\section{Basic Techniques: Code Walkthrough}
This walkthrough describes the functions of IllinoisGRMHD in the order
in which they are called. This order is specified in the {\tt
  schedule.ccl} file in the root directory of this thorn. The
scheduling goes as follows:

\textcolor{red}{WARNING: The scheduling description below is
  outdated, and based on the original version of IllinoisGRMHD, which
  depended on an ancient (ca. October 2010) version of ET. Scheduling
  is actually more straightforward now.}

\begin{itemize}

\item PRE-Initial Data ({\tt MoL\_Register} \& {\tt BASEGRID}) 
\begin{longtable}{|p{1.95in}|p{3.4in}|p{1.6in}|}
\hline
Function name & Description & Filename in {\tt src/} \\
\hline
{\tt IllinoisGRMHD\_RegisterVars} & Register variables
                   with the Method-of-Lines thorn & {\tt
                     MoL\_registration.C} \\
{\tt IllinoisGRMHD\_InitSymBound} & Specify variable symmetry properties, so that simulations with assumed
    symmetries may be performed. E.g., when assuming bitant
    symmetry across the $xy-$plane, $v^z$ above the plane must be
    negative that below the plane, so that $v^z\equiv0$ in the
    plane. & {\tt InitSymBound.C} \\
\hline
\end{longtable}

\item POST-Initial Data, ({\tt ABE\_PostInitial}, inside {\tt
  CCTK\_POSTPOSTINITIAL})
\begin{longtable}{|p{1.95in}|p{3.4in}|p{1.6in}|}
\hline
Function name & Description & Filename in {\tt src/} \\
\hline
{\tt driver\_prims\_to\_conserv} (FIXME:
currently disabled) &  The initial data routine {\it should not} need
to set the conservative variables. Instead it should focus on, e.g.,
pressure, density, velocity, etc. This routine would compute
conservatives based on these primitives. & {\tt
      driver\_prims\_to\_conserv.C} \\

{\tt IllinoisGRMHD\_PostInitialData} & This routine sets
    symmetry ghostzones on all conservative \& primitive variables,
    then (FIXME: probably unnecessary) sets {\tt \_p} and {\tt \_p\_p}
    timelevels & {\tt postinitialdata.C} \\
\hline
\end{longtable}

\item Evolution Step (inside {\tt CCTK\_EVOL} schedule bin)
\begin{longtable}{|p{1.95in}|p{3.4in}|p{1.6in}|}
\hline
Function name & Description & Filename in {\tt src/} \\
\hline
{\tt driver\_evaluate\_MHD\_rhs} &
Evaluate RHSs of ~Eqs.~\ref{eq:Continuity}, \ref{eq:Energy}, \ref{eq:Euler},
\ref{eq:AInduction} and \ref{eq:EMGauge}, using original PPM
reconstruction and staggered vector potential variables. & {\tt
  driver\_evaluate\_MHD\_rhs.C} \\
\hline
\end{longtable}

\pagebreak
\item Outer Boundary Update ({\tt ABE\_PostStep}, right after {\tt
MoL\_PostStep} in {\tt CCTK\_EVOL})
\begin{longtable}{|p{1.95in}|p{2.8in}|p{2.2in}|}
\hline
Function name & Description & Filename in {\tt src/} \\
\hline
{\tt apply\_frozen\_boundary\_condition} & Apply frozen boundary
conditions (BCs), if MATTER\_BC and EM\_BC are set to 2 (FROZEN). Note that this is the
ONLY BC option at applies BCs directly to the conservative variables!
All other BCs should be scheduled AFTER the primitives are solved
for. & {\tt apply\_frozen\_boundary\_condition.C} \\

{\tt compute\_B\_and\_Bstagger\_from\_A} & The easiest primitive
variables to solve for: compute $B^i$ from $A_i$, via simple finite
differences. We also compute $B^i$ on staggered gridpoints as
well. & {\tt compute\_B\_and\_Bstagger\_from\_A.C} \\

{\tt conserv\_to\_prims} & The conservative variables depend
nonlinearly on the primitive variables, requiring us in general to
solve for the primitive variables via a Newton-Raphson root finder. We
employ a modified version of the very nice, open-sourced HARM-based
conserv-to-prim solver of Noble et al. & {\tt
  driver\_conserv\_to\_prims.C} \\

FIXME: Insert call to ET boundary condition driver, to apply BCs to
all primitive variables when BCs are set to something other than
FROZEN. & & \\

FIXME: Insert call to compute conservative variables from primitive
variables on outer boundaries only. & & \\
\hline
\end{longtable}
\item We call the following functions after restriction, after
  prolongation, and after timelevels are switched (since primitives
  are only stored on one timelevel). The fundamental issue is, in
  order to evaluate the RHS of the GRMHD equations, we need primitives
  to be defined at ALL gridpoints AND be conistent with conservative
  variables. Restriction \& prolongation updates conservative
  variables, and after timelevels are switched, primitive variables
  (defined only on one timelevel) become inconsistent with
  conservatives. This schedule bin fixes these issues.
\begin{longtable}{|p{1.95in}|p{2.8in}|p{2.2in}|}
\hline
Function name & Description & Filename in {\tt src/} \\
\hline
{\tt compute\_B\_and\_Bstagger\_from\_A} & See description above. &
{\tt compute\_B\_and\_Bstagger\_from\_A.C} \\
{\tt conserv\_to\_prims} & See description above. & {\tt
  driver\_conserv\_to\_prims.C} \\
\hline
\end{longtable}
\end{itemize}


\subsection{{\tt driver\_evaluate\_MHD\_rhs()}}

\subsubsection{Fill basic data structures and initialize GRMHD RHS variables}
First you will see that we set a struct called {\tt eos}, short for
Equation Of State. This is an easily extensible data structure that
stores all the required information for the equation of state, which
closes the set of GRMHD equations. We must store the parameters that
populate this struct as separate variables in {\tt param.ccl} and {\tt
  interface.ccl}, since, to my knowledge, ET does not support defining
{\tt struct}'s in these files. (FIXME).

Next you will notice that we declare {\tt in\_prims}, {\tt out\_prims\_r},
{\tt out\_prims\_l}, which are arrays of the {\tt gf\_and\_gz\_struct}
type of {\tt struct}. {\tt gf\_and\_gz\_struct} is very simply defined as 
\begin{verbatim}
struct gf_and_gz_struct {
  double *gf;
  int gz_lo[4],gz_hi[4];
}
\end{verbatim}
This is a structure that simply points to the address in memory of
the gridfunction, and stores the ghostzone information for that
gridfunction. Ghostzone information comes in very handy when we need
to reconstruct twice, as is the case when evaluating the induction
equation on staggered grids.

FIXME: Next we set the gridfunction $\psi=e^{\phi}$. Really this
should be done by the BSSN thorn (i.e., the thorn that handles the
metric evolution).

We then set a couple of arrays of pointers, that simply point to the
address in memory of metric-related and $T^{\mu\nu}$
gridfunctions. Technically we do not need to store $T^{\mu\nu}$ in
memory, since it can be recomputed. However, the computational cost is
not very low, and this code assumes, to an extent, that memory space
is cheap. For cases in which you are running too low on memory,
you may want to rewrite this code to remove $T^{\mu\nu}$. There are
several other optimizations you could make that reduce the memory cost
at the price of more memory accesses and higher CPU cost.

Finally, we set the RHSs of all conservative variables to zero.

\subsubsection{{\tt compute\_tau\_rhs\_curvature\_terms\_and\_TUPmunu}}
Eq.~\ref{tau_curvature} contains the only term on the RHS of the GRMHD
equations that depends only on $T^{\mu\nu}$ and metric quantities without
derivatives. Now, $T^{\mu\nu}$ will come in handy a bit later when we
actually need to compute the metric source terms for $\tilde{S}_i$,
and the rest of Eq.~\ref{tau_curvature} which depend on derivatives of
the metric.

\subsubsection{{\tt ftilde\_gf\_compute}}
Next we specify the first flux direction ($x$=1), and then compute the
function $\tilde{f}$ used in the Piecewise Parabolic Method
(PPM). This function is used for flattening face values around shocks.

\subsubsection{{\tt reconstruct\_set\_of\_prims\_PPM}}
This is the master function for applying PPM reconstruction to a set
of variables. All hydro primitive variables ($\rho_b,P,v^i$) are
defined at grid vertices, yet in the GRMHD evolution equations, we
must compute fluxes of these quantities. Evaluating these fluxes
requires that divergences be computed--a task ideally suited for a
simple finite difference approach, or so one might think. In reality,
when characteristics cross and the resulting shocks develop into
discontinuities, simple finite differences across shocks results in
the Gibbs phenomenon, in which approximating these derivatives with
simple, overlapping polynomials results in wild oscillations around
physical discontinuities. We get around this problem via a more
sophisticated algorithm, as follows:

First, we very carefully computing the flux in the $x$-direction by
{\it reconstructing} primitive variables, which are defined at
$(i,j,k)$, to the cell {\it face} at $(i+\frac{1}{2},j,k)$. We employ
the standard PPM reconstruction scheme of Collela and Woodward in this
code, modified so that it depends on only 3 ghostzones (the
4-ghostzone version includes better shock detection). This scheme,
which is third-order accurate in resolution for smooth flows and
first-order for shocks, effectively reduces spurious oscillations from
developing across shocks. 

\subsubsection{{\tt add\_fluxes\_and\_source\_terms\_to\_hydro\_rhss}}
After reconstructing all the GRMHD primitive variables in a single
direction, we are then ready to evaluate the flux terms on the RHSs of the GR HYDRO
equations. Before doing that, this routine starts by evaluating the
remaining source terms for $\tau$ and $\tilde{S}_i$, then it calls the
function that computes the flux terms, called {\tt mhdflux}.

{\tt mhdflux()} (located in {\tt src/mhdflux.C} takes the reconstructed
GRMHD primitive variables and uses them to evaluate the flux terms on
the RHSs of the GR HYDRO equations. It starts by evaluating the
enthalpy $h$ and the internal energy $\epsilon$ based on the equation
of state. Then it computes fluxes and approximate characteristic
speeds for all the conservative quantities. Note that these speeds
overestimate the actual characteristic speeds, which results in a more
dissipative (but potentially more stable) algorithm. Contributing
further to the dissipation, we compute fluxes at face values via the
original HLL scheme, which is known to be overly dissipative. There
are better schemes, such as HLLC or HLLD (Athena code uses) (FIXME),
but the goal of this thorn release is that it yield results that are
to roundoff precision identical to those obtained from the original
GRMHD code of the Illinois NR group.

After {\tt mhdflux()} is called, we then compute the GR HYDRO RHS
quantities via simple finite differences, so that, in the case of the
$x$-direction, the RHS flux term is simply $(F_{i+1/2,j,k} -
F_{i-1/2,j,k})/dx$. 

Evaluation of $\rho_*,\tilde{S}_i,$ and $\tau$ RHS's in the $y$ and
$z$-directions follows trivially from how it is done in the
$x$-direction. However, evaluating the RHSs of the induction
equations, while straightforward, requires quite a bit more
bookkeeping, which is described in the next section.

\subsubsection{{\tt A\_i\_rhs\_no\_gauge\_terms()}: Evaluating
  $\epsilon_{ijk} v^j B^k$ term on RHS of $\partial_t A_i$}
\label{A_i_rhs_no_gauge_terms}
Unlike the primitive variables ($\rho_b,P,v^i,B^i$), which are defined
at vertex centers ($i,j,k$), $A_{\mu}$'s are defined at staggered
gridpoints. In particular, they are defined as follows:

\begin{center}
\begin{tabular}{cc}
\hline
  Variable & storage location \\
\hline
  $\psi^6\Phi$ & $(i+\half,j+\half,k+\half)$ \\
  $A_x$ & $(i,j+\half,k+\half)$ \\
  $A_y$ & $(i+\half,j,k+\half)$ \\
  $A_z$ & $(i+\half,j+\half,k)$ \\
\hline
\label{staggered_A_mu_table}
\end{tabular}
\end{center}

{\tt A\_i\_rhs\_no\_gauge\_terms()} computes the first part of Eq.~(\ref{eq:AInduction}),
namely: $\epsilon_{ijk} v^j B^k$, where $\epsilon_{ijk}=\sqrt{\gamma}
\tilde{\epsilon}_{ijk}$, where $\tilde{\epsilon}_{ijk}$ is the flat
space Levi-Civita operator and $\sqrt{\gamma}\equiv \psi^6 \equiv e^{6
  \phi}$ . So in the case of $A_z$, we have
\beq
\epsilon_{xjk} v^j B^k = \sqrt{\gamma} ( v^x B^y - v^y B^x ).
\label{eq:AInduction_firstpart}
\eeq
Here is our strategy for computing this term:

\begin{enumerate}
\item Since $A_z$ is staggered on points ($i+1/2,j+1/2,k$), we first
  interpolate $\phi=\log(\psi)$ to these staggered points (done inside
  {\tt driver\_evaluate\_MHD\_rhs.C}). 
\item Next, inside {\tt driver\_evaluate\_MHD\_rhs.C}, we reconstruct
  $B^x$ and $B^y$ so that they are defined at staggered points
  ($i+1/2,j+1/2,k$). To do this, we first note that the gridfunction 
  {\tt Bx\_stagger} is defined at ($i+1/2,j,k$) and {\tt Bz\_stagger}
  at ($i,j,k+1/2$) (similarly, {\tt   By\_stagger} is defined at
  ($i,j+1/2,k$).
  \begin{enumerate}
  \item So to get {\tt Bx\_stagger} at the needed ($i+1/2,j+1/2,k$),
    we reconstruct it along the $y$-direction to ($i+1/2,j-1/2,k$) and
    use a simple $j\to j+1$ shift.
    
  \item Similarly, to get {\tt By\_stagger} at the needed
    ($i+1/2,j+1/2,k$), we reconstruct it along the $x$-direction to
    ($i-1/2,j+1/2,k$) and use a simple $i\to i+1$ shift.
  \end{enumerate}
\item Finally, $v^y$ and $v^z$ are defined at ($i,j,k$). So to get them to
  the needed ($i+1/2,j+1/2,k$), we reconstruct first along $x$ to
  ($i-1/2,j,k$) and then along $y$, which gets us to
  ($i-1/2,j-1/2,k$). Then we use a simple $i\to i+1$ and $j\to j+1$
  shift, which brings us to ($i+1/2,j+1/2,k$).
\end{enumerate}

If we simply followed these steps for computing
$\epsilon_{ijk} v^j B^k$ in sequence of $i$, {\tt
  driver\_evaluate\_MHD\_rhs.C}) would certainly be much easier to
read. However, this would also result in a less efficient code, that
either uses more memory or makes more than the minimum number of calls
to the reconstruction routines (pick your poison). To remedy this, I
decided to comment the {\tt driver\_evaluate\_MHD\_rhs()}) function
extensively, adding many bookkeeping notes.

\subsubsection{{\tt
    Lorenz\_psi6phi\_rhs\_\_add\_gauge\_terms\_to\_A\_i\_rhs()}:
  Evaluating $(\psi^6 \Phi)$ RHS (damped Lorenz gauge) and $\partial_i
  (\alpha \Phi -\beta^j A_j)$ terms in $A_i$ RHS}

% A bit pedantic...
%First a word on the naming of this function. We call $\partial_i
%  (\alpha \Phi -\beta^j A_j)$ the ``gauge terms of $A_i$'', since
%gauge freedom allows us to set it to zero if we like. This would yield
%the dreaded ``algebraic'' gauge, which possesses non-propagating gauge
%modes that can result in the buildup of spurious gauge {\it and
%  physical} modes on AMR refinement boundaries, left there by
%truncation errors in the prolongation step. 

\begin{center}
\end{center}
At the end of {\tt driver\_evaluate\_MHD\_rhs()}), we call the
function {\tt Lorenz\_psi6phi\_rhs\_\_add\_gauge\_terms\_to\_A\_i\_rhs()}, which
adds the $\partial_i (\alpha \Phi -\beta^j A_j)$ terms to the RHS of
the $A_i$ evolution equations (Eq.~\ref{eq:AInduction}) and evaluates
the RHS of the $(\psi^6 \Phi)$ evolution equation
(Eq.~\ref{eq:EMGauge}). In particular, this function evaluates:

\beqn
&& - \partial_j (\alpha \sqrt{\gamma}\, A^j - \sqrt{\gamma}\, \beta^j \Phi)
- \xi \alpha (\sqrt{\gamma} \Phi) \text{ entire RHS of $(\psi^6 \Phi)$
evolution equation (Eq.~\ref{eq:EMGauge}), and} 
\label{psi6phiRHS}
 \\
&& - \partial_i (\alpha \Phi -\beta^j A_j) \text{ remainder of $A_i$
  evolution equation RHS (Eq.~\ref{eq:AInduction}).}
\label{AiRHS_gauge_terms}
\eeqn

As was the case with the $\epsilon_{ijk} v^j B^k$ terms on the RHS of
the $A_i$ evolution equations, the subtlety in evaluating the RHS of
$\psi^6 \Phi$ (expression~\ref{psi6phiRHS}) lies completely in the
fact that $(\psi^6\Phi)$ is defined on the staggered grid
($i+1/2,j+1/2,k+1/2$), and this subtletly is compounded by the facts
that 
\begin{enumerate}
\item the $A_i$'s in this expression also lie on staggered grids (see
Section~\ref{A_i_rhs_no_gauge_terms} for full details), and 
\item the terms~(\ref{psi6phiRHS}) contain derivatives! 
\end{enumerate}
To know which functions to interpolate to which
gridpoints, we use the following map for the $x$-term in the RHS
gradient term:

\beqn
\partial_t (\psi^6 \Phi)_X
&=& - \partial_x [\alpha \sqrt{\gamma} A^x -
\beta^x (\sqrt{\gamma} \Phi)] - \xi \alpha (\sqrt{\gamma} \Phi) \\
&=& - \partial_x [\beta^x (\psi^6\Phi)]
- \partial_x [\alpha \sqrt{\gamma} A^x]
- \xi \alpha (\psi^6\Phi) \\
&& \uparrow \text{ exact analytical expressions } \uparrow \nonumber \\
\hline
&& \downarrow \text{ numerical approximation } \downarrow \nonumber \\
&=& 
- U_x [\beta^x (\psi^6\Phi)]
- C_x [\alpha \psi^6 A^x]
- \xi \alpha (\psi^6\Phi) \\
&=& 
- U_x [a(\beta^x,(\psi^6\Phi))]
- C_x [b(\alpha,\psi^6,A^x)]
- \xi \alpha (\psi^6\Phi) \\
\eeqn
where $U_x$ and $C_x$ indicates an upwinded or centered finite
difference stencil, respectively. More explicitly, $U_x a$ is given by
\beq 
(2 \Delta x) U_x a(\beta^x,(\psi^6\Phi)) = \left\{ \begin{array}{ll}
  a_{i-3/2,j+1/2,k+1/2} - 4 a_{i-1/2,j+1/2,k+1/2} + 3 a_{i+1/2,j+1/2,k+1/2} & \mbox{if $\beta^x < 0$\ ,} \\
  -a_{i+5/2,j+1/2,k+1/2} + 4 a_{i+3/2,j+1/2,k+1/2}- 3 a_{i+1/2,j+1/2,k+1/2} & \mbox{otherwise\ .}
\label{psi6phi_firstterm}
\end{array} \right .
\eeq
Also, $C_x$ is given by
\beq
(\Delta x) C_x b = b_{i+1,j+1/2,k+1/2} - b_{i,j+1/2,k+1/2}, b=\alpha \psi^6 A^x.
\label{psi6phi_secondterm}
\eeq

Using this map as our guide, we now review the code
step-by-step. Computing the $y$ and $z$ components are
straightforward upon understanding the $x$ component evaluation.
\begin{enumerate}
\item Eq.~(\ref{psi6phi_secondterm}) tells us the $x$-component of the gradient on the
RHS of $\psi^6\Phi$ depends on $\alpha,\psi^6,$ and $A^x$ at the point
($i,j+1/2,k+1/2$). However, we do not store $A^x$ as a gridfunction in
the code, or even $\psi^6$ for that matter, so they must be first
interpolated to ($i,j+1/2,k+1/2$). Note that 
\beqn
\alpha \psi^6 A^x &=& \alpha \psi^6 g^{xj} A_j \\
&=& \alpha \psi^2 \tilde{\gamma}^{xj} A_j
\eeqn
We first multiply $\alpha \psi^2$ and save it as a separate
gridfunction, then interpolate $\alpha\psi^2$ and $\tilde{\gamma}^{xj}$ to
($i,j+1/2,k+1/2$). Interpolation is performed at second order accuracy
through simple averaging, then the derivative is computed as specified
in Eq.~(\ref{psi6phi_secondterm}).

FIXME: We should really unstagger $A_x$, then raise it, and then do
interpolations on $A^x$. This would yield a more efficient algorithm.

\item Next we compute the damping term $-\xi \alpha (\psi^6\Phi)$,
  which requires only that $\alpha$ is interpolated to
  ($i+1/2,j+1/2,k+1/2$) via simple averaging.

\item Then we evaluate the expression~(\ref{AiRHS_gauge_terms}):
$\partial_x (\alpha \Phi -\beta^j A_j)$, which goes on the RHS of the
$A_x$ evolution equation. We do this by first interpolating $(\alpha \Phi
-\beta^j A_j)$ to ($i+1/2,j+1/2,k+1/2$).  Then we evaluate the
derivative as follows
\beqn
\partial_x (\alpha \Phi -\beta^j A_j) &=& \partial_x d \\
&=& (d_{i+1/2,j+1/2,k+1/2} - d_{i-1/2,j+1/2,k+1/2})/(\Delta x)
\eeqn
This term is defined at ($i,j+1/2,k+1/2$), consistent with the RHS of
$\partial_t A_x$, where this term is added.

\item Finally we evaluate Eq.~(\ref{psi6phi_firstterm}). There is
  nothing special here; we need only interpolate $\beta^i$ to ($i+1/2,j+1/2,k+1/2$).
\end{enumerate}


\end{document}
