% *======================================================================*
%  Cactus Thorn template for ThornGuide documentation
%  Author: Ian Kelley
%  Date: Sun Jun 02, 2002
%  $Header$
%
%  Thorn documentation in the latex file doc/documentation.tex
%  will be included in ThornGuides built with the Cactus make system.
%  The scripts employed by the make system automatically include
%  pages about variables, parameters and scheduling parsed from the
%  relevant thorn CCL files.
%
%  This template contains guidelines which help to assure that your
%  documentation will be correctly added to ThornGuides. More
%  information is available in the Cactus UsersGuide.
%
%  Guidelines:
%   - Do not change anything before the line
%       % START CACTUS THORNGUIDE",
%     except for filling in the title, author, date, etc. fields.
%        - Each of these fields should only be on ONE line.
%        - Author names should be separated with a \\ or a comma.
%   - You can define your own macros, but they must appear after
%     the START CACTUS THORNGUIDE line, and must not redefine standard
%     latex commands.
%   - To avoid name clashes with other thorns, 'labels', 'citations',
%     'references', and 'image' names should conform to the following
%     convention:
%       ARRANGEMENT_THORN_LABEL
%     For example, an image wave.eps in the arrangement CactusWave and
%     thorn WaveToyC should be renamed to CactusWave_WaveToyC_wave.eps
%   - Graphics should only be included using the graphicx package.
%     More specifically, with the "\includegraphics" command.  Do
%     not specify any graphic file extensions in your .tex file. This
%     will allow us to create a PDF version of the ThornGuide
%     via pdflatex.
%   - References should be included with the latex "\bibitem" command.
%   - Use \begin{abstract}...\end{abstract} instead of \abstract{...}
%   - Do not use \appendix, instead include any appendices you need as
%     standard sections.
%   - For the benefit of our Perl scripts, and for future extensions,
%     please use simple latex.
%
% *======================================================================*
%
% Example of including a graphic image:
%    \begin{figure}[ht]
% 	\begin{center}
%    	   \includegraphics[width=6cm]{MyArrangement_MyThorn_MyFigure}
% 	\end{center}
% 	\caption{Illustration of this and that}
% 	\label{MyArrangement_MyThorn_MyLabel}
%    \end{figure}
%
% Example of using a label:
%   \label{MyArrangement_MyThorn_MyLabel}
%
% Example of a citation:
%    \cite{MyArrangement_MyThorn_Author99}
%
% Example of including a reference
%   \bibitem{MyArrangement_MyThorn_Author99}
%   {J. Author, {\em The Title of the Book, Journal, or periodical}, 1 (1999),
%   1--16. {\tt http://www.nowhere.com/}}
%
% *======================================================================*

% If you are using CVS use this line to give version information
% $Header$

\documentclass{article}

% Use the Cactus ThornGuide style file
% (Automatically used from Cactus distribution, if you have a
%  thorn without the Cactus Flesh download this from the Cactus
%  homepage at www.cactuscode.org)
\usepackage{../../../../doc/latex/cactus}

\begin{document}

% The author of the documentation
\author{Zachariah B.~Etienne \textless zachetie *at* gmail *dot* com\textgreater}

% The title of the document (not necessarily the name of the Thorn)
\title{\texttt{BaikalVacuum}: An Einstein Toolkit thorn for solving Einstein's
  equations in vacuum (e.g., black hole) spacetimes, in the BSSN
  formalism, in Cartesian coordinates}

% the date your document was last changed, if your document is in CVS,
% please use:
%    \date{$ $Date$ $}
\date{May 18, 2020}

\maketitle

% Do not delete next line
% START CACTUS THORNGUIDE

% Add all definitions used in this documentation here
%   \def\mydef etc
\newcommand{\beq}{\begin{equation}}
\newcommand{\eeq}{\end{equation}}
\newcommand{\beqn}{\begin{eqnarray}}
\newcommand{\eeqn}{\end{eqnarray}}
\newcommand{\half} {{1\over 2}}
\newcommand{\sgam} {\sqrt{\gamma}}
\newcommand{\tB}{\tilde{B}}
\newcommand{\tS}{\tilde{S}}
\newcommand{\tF}{\tilde{F}}
\newcommand{\tf}{\tilde{f}}
\newcommand{\tT}{\tilde{T}}
\newcommand{\sg}{\sqrt{\gamma}\,}
\newcommand{\ve}[1]{\mbox{\boldmath $#1$}}

\newcommand{\GiR}{{\texttt{GiRaFFE}}}
\newcommand{\IGM}{{\texttt{IllinoisGRMHD}}}

\linespread{1.0}

\newenvironment{packed_itemize}{
\begin{itemize}
  \setlength{\itemsep}{0.0pt}
  \setlength{\parskip}{0.0pt}
  \setlength{\parsep}{ 0.0pt}
}{\end{itemize}}

\newenvironment{packed_enumerate}{
\begin{enumerate}
  \setlength{\itemsep}{0.0pt}
  \setlength{\parskip}{0.0pt}
  \setlength{\parsep}{ 0.0pt}
}{\end{enumerate}}

% Add an abstract for this thorn's documentation
\begin{abstract}
\texttt{BaikalVacuum} solves Einstein's equations of general
relativity in the BSSN formalism, for the case of vacuum (e.g., black
hole) spacetimes. In such spacetimes Einstein's equations reduce to
$G^{\mu \nu}=0$. Given that the complications and
super-short-lengthscale variations usually associated with the
stress-energy tensor $T^{\mu\nu}$ in strongly curved spacetimes do not
exist in vacuum spacetimes, this thorn implements very high order
finite-difference derivatives up to and including 8th order. All codes
within this thorn were generated using the Python-based NRPy+
infrastructure, which may be found at \url{http://nrpyplus.net/} (alt
link: \url{https://github.com/zachetienne/nrpytutorial}).
\end{abstract}

Both \texttt{BaikalVacuum} and its sister thorn \texttt{Baikal} are
fully documented in Jupyter notebooks within NRPy+, and
the latest documentation for \texttt{BaikalETK} (the name of the
combined \texttt{Baikal} / \texttt{BaikalVacuum} project) may be found
in the \texttt{Tutorial-ETK\_thorn-BaikalETK.ipynb} Jupyter notebook here:

\url{https://github.com/zachetienne/nrpytutorial}

Also the NRPy+ Jupyter documentation can be navigated via the
following link:

\url{https://nbviewer.jupyter.org/github/zachetienne/nrpytutorial/blob/master/Tutorial-ETK_thorn-BaikalETK.ipynb}

\section{Planned features}

\begin{itemize}
\item Currently \texttt{Baikal} / \texttt{BaikalVacuum} do not officially
support numerical grids containing symmetry assumptions, but
appropriate symmetries should be registered. Users are encouraged to
proceed with caution when using symmetries. Future work will include adding
official support for symmetries.
\item \texttt{Baikal} / \texttt{BaikalVacuum} do not yet support an
  extensive variety of gauge conditions, instead defaulting to the
  tried-and-true standard moving puncture gauge conditions ($1+\log$
  lapse combined with the ``Gamma''-driving shift [though note that
    \texttt{Baikal} / \texttt{BaikalVacuum} refer to the  ``Gamma''
    variable as the ``Lambda'' variable]). Additional gauge condition
  options will eventually be added.
\item \texttt{Baikal} / \texttt{BaikalVacuum} do not support
  \texttt{Llama}-based multiblock (e.g., cubed sphere) grids, and
  adding this support is currently not a high priority.
\end{itemize}



% Do not delete next line
% END CACTUS THORNGUIDE

\end{document}
