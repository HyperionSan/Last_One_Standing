% *======================================================================*
%  Cactus Thorn template for ThornGuide documentation
%  Author: Ian Kelley
%  Date: Sun Jun 02, 2002
%  $Header$
%
%  Thorn documentation in the latex file doc/documentation.tex
%  will be included in ThornGuides built with the Cactus make system.
%  The scripts employed by the make system automatically include
%  pages about variables, parameters and scheduling parsed from the
%  relevant thorn CCL files.
%
%  This template contains guidelines which help to assure that your
%  documentation will be correctly added to ThornGuides. More
%  information is available in the Cactus UsersGuide.
%
%  Guidelines:
%   - Do not change anything before the line
%       % START CACTUS THORNGUIDE",
%     except for filling in the title, author, date, etc. fields.
%        - Each of these fields should only be on ONE line.
%        - Author names should be separated with a \\ or a comma.
%   - You can define your own macros, but they must appear after
%     the START CACTUS THORNGUIDE line, and must not redefine standard
%     latex commands.
%   - To avoid name clashes with other thorns, 'labels', 'citations',
%     'references', and 'image' names should conform to the following
%     convention:
%       ARRANGEMENT_THORN_LABEL
%     For example, an image wave.eps in the arrangement CactusWave and
%     thorn WaveToyC should be renamed to CactusWave_WaveToyC_wave.eps
%   - Graphics should only be included using the graphicx package.
%     More specifically, with the "\includegraphics" command.  Do
%     not specify any graphic file extensions in your .tex file. This
%     will allow us to create a PDF version of the ThornGuide
%     via pdflatex.
%   - References should be included with the latex "\bibitem" command.
%   - Use \begin{abstract}...\end{abstract} instead of \abstract{...}
%   - Do not use \appendix, instead include any appendices you need as
%     standard sections.
%   - For the benefit of our Perl scripts, and for future extensions,
%     please use simple latex.
%
% *======================================================================*
%
% Example of including a graphic image:
%    \begin{figure}[ht]
% 	\begin{center}
%    	   \includegraphics[width=6cm]{MyArrangement_MyThorn_MyFigure}
% 	\end{center}
% 	\caption{Illustration of this and that}
% 	\label{MyArrangement_MyThorn_MyLabel}
%    \end{figure}
%
% Example of using a label:
%   \label{MyArrangement_MyThorn_MyLabel}
%
% Example of a citation:
%    \cite{MyArrangement_MyThorn_Author99}
%
% Example of including a reference
%   \bibitem{MyArrangement_MyThorn_Author99}
%   {J. Author, {\em The Title of the Book, Journal, or periodical}, 1 (1999),
%   1--16. {\tt http://www.nowhere.com/}}
%
% *======================================================================*

% If you are using CVS use this line to give version information
% $Header$

\documentclass{article}

% Use the Cactus ThornGuide style file
% (Automatically used from Cactus distribution, if you have a
%  thorn without the Cactus Flesh download this from the Cactus
%  homepage at www.cactuscode.org)
\usepackage{../../../../doc/latex/cactus}
%% \usepackage{cactus}

\begin{document}

% The author of the documentation
\author{Zachariah B.~Etienne \textless zachetie *at* gmail *dot* com \textgreater\\
  Leonardo R.~Werneck \textless wernecklr *at* gmail *dot* com \textgreater\\
  Ian Ruchlin \textless ianruchlin *at* gmail *dot* com \textgreater
}

% The title of the document (not necessarily the name of the Thorn)
\title{\texttt{VolumeIntegrals\_vacuum}: An Einstein Toolkit thorn for volume integrations in vacuum spacetimes}

% the date your document was last changed, if your document is in CVS,
% please use:
\date{May 20, 2021}

\maketitle

% Do not delete next line
% START CACTUS THORNGUIDE

% Add all definitions used in this documentation here
%   \def\mydef etc
\newcommand{\beq}{\begin{equation}}
\newcommand{\eeq}{\end{equation}}
\newcommand{\beqn}{\begin{eqnarray}}
\newcommand{\eeqn}{\end{eqnarray}}
\newcommand{\half} {{1\over 2}}
\newcommand{\sgam} {\sqrt{\gamma}}
\newcommand{\tB}{\tilde{B}}
\newcommand{\tS}{\tilde{S}}
\newcommand{\tF}{\tilde{F}}
\newcommand{\tf}{\tilde{f}}
\newcommand{\tT}{\tilde{T}}
\newcommand{\sg}{\sqrt{\gamma}\,}
\newcommand{\ve}[1]{\mbox{\boldmath $#1$}}
\newcommand{\Madm}{M_{\rm ADM}}
\newcommand{\Padm}{P_{\rm ADM}}
\newcommand{\Jadm}{J_{\rm ADM}}
\newcommand{\norm}[1]{\left\lVert#1\right\rVert}

\newcommand{\GiR}{{\texttt{GiRaFFE}}}
\newcommand{\IGM}{{\texttt{IllinoisGRMHD}}}

\linespread{1.0}

\newenvironment{packed_itemize}{
\begin{itemize}
  \setlength{\itemsep}{0.0pt}
  \setlength{\parskip}{0.0pt}
  \setlength{\parsep}{ 0.0pt}
}{\end{itemize}}

\newenvironment{packed_enumerate}{
\begin{enumerate}
  \setlength{\itemsep}{0.0pt}
  \setlength{\parskip}{0.0pt}
  \setlength{\parsep}{ 0.0pt}
}{\end{enumerate}}

% Add an abstract for this thorn's documentation
\begin{abstract}
  \texttt{VolumeIntegrals\_vacuum} is a thorn for integration of
  spacetime quantities, accepting integration volumes consisting of
  spherical shells, regions with hollowed balls, and simple spheres.
\end{abstract}

%%%%%%%%%%%%%%%%%%%%%%%%%%
%%%% Volume Integrals %%%%
%%%%%%%%%%%%%%%%%%%%%%%%%%
\section{Volume integrals}
\label{sec:volume_integrals}

We now briefly describe the volume integrals that can be performed
using the \texttt{VolumeIntegrals\_vacuum} thorn.

%%%%%%%%%%%%%%%%%%%%%%%%%%
%%%%%%%% L^2-norm %%%%%%%%
%%%%%%%%%%%%%%%%%%%%%%%%%%
\subsection{$L^{2}$-norm}
\label{sec:L2_norm_integral}

For a given field $f$, the $L^{2}$-norm of the field, $\norm{f}_{2}$,
is computed using the volume integral
%%
\beq
\boxed{\norm{f}_{2} = \int f^{2}dV}\; ,
\eeq
%%
where $dV$ is the volume element.

%%%%%%%%%%%%%%%%%%%%%%%%%%
%%%% Center of lapse %%%%%
%%%%%%%%%%%%%%%%%%%%%%%%%%
\subsection{Center of the lapse}
\label{sec:center_of_lapse}

The \emph{center of the lapse}, $C_{\alpha}$, volume integral yields
results which are pretty consistent with the center of mass volume
integral. We compute it using
%%
\beq
\boxed{ C_{\alpha}^{i} = \int \left[\left(1-\alpha\right)^{80}x^{i}\right]dV }\; ,
\eeq
%%
where $\alpha$ is the lapse function, $x^{i}$ is the position vector
and $dV$ is the volume element.

%%%%%%%%%%%%%%%%%%%%%%%%%%
%%%%%%%% ADM mass %%%%%%%%
%%%%%%%%%%%%%%%%%%%%%%%%%%
\subsection{ADM mass}
\label{sec:ADM_mass_integral}

The ADM mass, $\Madm$ is computed using equation (A.5) in~\cite{alcubierre2008introduction}
(see also eq. (7.15) in~\cite{gourgoulhon20073+})
%%
\beq
\Madm = \frac{1}{16\pi}\lim_{r\to\infty}\oint_{S}\left[
  \delta^{ij}\left(\partial_{i}h_{jk} - \partial_{k}h_{ij}\right)dS^{k}
  \right],
\eeq
%%
where $S$ is the surface of integration and $dS^{i} = s^{i}dA$, with
$s^{i}$ the unit outward-pointing normal vector to the surface and
$dA$ the area element. To obtain the equation above, the physical
spatial metric, $\gamma_{ij}$, has been decomposed using
%%
\beq
\gamma_{ij} = \delta_{ij} + h_{ij},
\eeq
%%
where $\delta_{ij}$ is the Kronecker delta and represents the flat
space metric in Cartesian coordinates, while $h_{ij}$ is a small
perturbation around flat space physical spatial metric. In practice,
we do not take the integration surface to be at infinity, and therefore
we implement the expression
%%
\beq
\boxed{
\Madm = \frac{1}{16\pi}\oint_{S}\left[
  \gamma^{ij}\left(\partial_{i}\gamma_{jk} - \partial_{k}\gamma_{ij}\right)dS^{k}
  \right]
}\; .
\eeq
%%
where $\gamma^{ij}$ is the inverse spatial metric.

%%%%%%%%%%%%%%%%%%%%%%%%%%
%%%%%% ADM momentum %%%%%%
%%%%%%%%%%%%%%%%%%%%%%%%%%
\subsection{ADM momentum}
\label{sec:ADM_momentum_integral}

The ADM momentum, $\Padm^{i}$, is obtained from equation (A.6) in
\cite{alcubierre2008introduction} (see also eq. (7.56) in
\cite{gourgoulhon20073+})
%%
\beq
\Padm^{i} = \frac{1}{8\pi}\lim_{r\to\infty}\oint_{S}\left[
  \left(K^{i}_{\ j}-\delta^{i}_{\ j}K\right)dS^{j}
  \right],
\eeq
%%
where $K_{ij}$ is the extrinsic curvature and $K\equiv\gamma^{ij}K_{ij}$
is the mean curvature. Like the ADM mass, the integration is not performed
at infinity, and the implemented equation is
%%
\beq
\boxed{
\Padm^{i} = \frac{1}{8\pi}\oint_{S}\left[
  \left(K^{ij}-\gamma^{ij}K\right)dS_{j}
  \right]
}\; .
\eeq
%%

%%%%%%%%%%%%%%%%%%%%%%%%%%
%% ADM angular momentum %%
%%%%%%%%%%%%%%%%%%%%%%%%%%
\subsection{ADM angular momentum}
\label{sec:ADM_angular_momentum_integral}

The ADM angular momentum, $\Jadm^{i}$, follows from equation (A.7) in
\cite{alcubierre2008introduction} (see also eq. (7.63) in
\cite{gourgoulhon20073+})
%%
\beq
\Jadm^{i} = \frac{1}{8\pi}\lim_{r\to\infty}\oint_{S}\left[
  \epsilon^{ijk}x_{j}\left(K_{kl} - \delta_{kl}K\right)dS^{l}
  \right],
\eeq
%%
where $\epsilon^{ijk}$ is the three-dimensional Levi-Civita tensor and
$x^{i}$ are the components of the position vector in Cartesian coordinates.
At a finite distance from the origin this equation is written as
%%
\beq
\boxed{
\Jadm^{i} = \frac{1}{8\pi}\oint_{S}\left[
  \epsilon^{ijk}x_{j}\left(K_{kl} - \gamma_{kl}K\right)dS^{l}
  \right]
}\; .
\eeq
%%


%%%%%%%%%%%%%%%%%%%%%%%%%%
%%%%%% Basic usage %%%%%%%
%%%%%%%%%%%%%%%%%%%%%%%%%%
\section{Basic usage}
\label{sec:basic_usage}

Except for the definition of the integrands, the behavior of this thorn
is almost completely driver by the configuration of the parameter file.
We present here an example of such a parameter file, which performs the
following tasks:

%%%%%%%%%%%%%%%%%%%%%%%%%%
%%%% Evolution thorns %%%%
%%%%%%%%%%%%%%%%%%%%%%%%%%
\subsection{Specifying the BSSN evolution thorn}
\label{sec:evolution_thorn}

The \texttt{VolumeIntegrals\_vacuum} thorn requires information about the
Hamiltonian and momentum constraint variables in order to perform certain
volume integrals. For maximum flexibility, one can specify which variables
to use, making \texttt{VolumeIntegrals\_vacuum} compatible with any BSSN
evolution thorn. This is achieved by setting the following variables:
%%
\begin{packed_enumerate}
\item \verb|VolumeIntegrals_vacuum::HamiltonianVarString|;
\item \verb|VolumeIntegrals_vacuum::Momentum0VarString|;
\item \verb|VolumeIntegrals_vacuum::Momentum1VarString|;
\item \verb|VolumeIntegrals_vacuum::Momentum2VarString|;
\end{packed_enumerate}
%%
The default values for these variables are the variables from the
\texttt{ML\_BSSN} thorn, but you can use any thorn you want. For example,
to use the variables \verb|Ham|, \verb|MU0|, \verb|MU1|, and \verb|MU2| from
an evolution thorn called \texttt{MyBSSNthorn}, one would add the following
lines to the parameter file:
%%
\begin{verbatim}
VolumeIntegrals_vacuum::HamiltonianVarString = "MyBSSNthorn::Ham"
VolumeIntegrals_vacuum::Momentum0VarString   = "MyBSSNthorn::MU0"
VolumeIntegrals_vacuum::Momentum1VarString   = "MyBSSNthorn::MU1"
VolumeIntegrals_vacuum::Momentum2VarString   = "MyBSSNthorn::MU2"
\end{verbatim}
%%

%%%%%%%%%%%%%%%%%%%%%%%%%%
%%%%%% Full example %%%%%%
%%%%%%%%%%%%%%%%%%%%%%%%%%
\subsection{Full parameter file configuration example}
\label{sec:full_parfileexample}

We now provide an example of a parameter file configuration which uses
the Hamiltonian and momentum constraint variables of the \texttt{Baikal}
thorn and performs the following tasks:
%%
\begin{packed_enumerate}
\item Integral of Hamiltonian \& momentum constraints, over the entire
  grid volume.
  \label{int1}
\item Exactly the same as~\ref{int1}, but excising the region inside a
  sphere of radius 2.2 tracking the zeroth AMR grid (initially at
  $(x,y,z) = (-5.9,0,0)$);
  \label{int2}
\item Exactly the same as \ref{int2}, but additionally excising the
  region inside a sphere of radius 2.2 tracking the center of the first
  AMR grid (initially at $(x,y,z)=(+5.9,0,0)$);
  \label{int3}
\item Integral of Hamiltonian \& momentum constraints, over the entire
  grid volume, minus the spherical region inside coordinate radius 8.2;
  \label{int4}
\item Integral of Hamiltonian \& momentum constraints, over the entire
  grid volume, minus the spherical region inside coordinate radius 12.0;
  \label{int5}
\item Integral of Hamiltonian \& momentum constraints, over the entire
  grid volume, minus the spherical region inside coordinate radius 16.0;
  \label{int6}
\item Integral of Hamiltonian \& momentum constraints, over the entire
  grid volume, minus the spherical region inside coordinate radius 24.0;
  \label{int7}
\item Integral of Hamiltonian \& momentum constraints, over the entire
  grid volume, minus the spherical region inside coordinate radius 48.0;
  \label{int8}
\item Integral of Hamiltonian \& momentum constraints, over the entire
  grid volume, minus the spherical region inside coordinate radius 96.0;
  \label{int9}
\item Integral of Hamiltonian \& momentum constraints, over the entire
  grid volume, minus the spherical region inside coordinate radius 192.0;
  \label{int10}
\end{packed_enumerate}
%%
To achieve this, add the following configuration to your parameter file:
%%
\begin{verbatim}
ActiveThorns = "VolumeIntegrals_vacuum"
# Set the Hamiltonian and momentum constraint variables to Baikal's
VolumeIntegrals_vacuum::HamiltonianVarString = "Baikal::HGF"
VolumeIntegrals_vacuum::Momentum0VarString   = "Baikal::MU0GF"
VolumeIntegrals_vacuum::Momentum1VarString   = "Baikal::MU1GF"
VolumeIntegrals_vacuum::Momentum2VarString   = "Baikal::MU2GF"

# Now setup basic information about the integrals
VolumeIntegrals_vacuum::NumIntegrals = 10
VolumeIntegrals_vacuum::VolIntegral_out_every = 64
VolumeIntegrals_vacuum::enable_file_output = 1
VolumeIntegrals_vacuum::outVolIntegral_dir = "volume_integration"
VolumeIntegrals_vacuum::verbose = 1
# The AMR centre will only track the first referenced integration
# quantities that track said centre. Thus, centeroflapse output will
# not feed back into the AMR centre positions.
VolumeIntegrals_vacuum::Integration_quantity_keyword[1] = "H_M_CnstraintsL2"
VolumeIntegrals_vacuum::Integration_quantity_keyword[2] = "usepreviousintegrands"
VolumeIntegrals_vacuum::Integration_quantity_keyword[3] = "usepreviousintegrands"
VolumeIntegrals_vacuum::Integration_quantity_keyword[4] = "H_M_CnstraintsL2"
VolumeIntegrals_vacuum::Integration_quantity_keyword[5] = "H_M_CnstraintsL2"
VolumeIntegrals_vacuum::Integration_quantity_keyword[6] = "H_M_CnstraintsL2"
VolumeIntegrals_vacuum::Integration_quantity_keyword[7] = "H_M_CnstraintsL2"
VolumeIntegrals_vacuum::Integration_quantity_keyword[8] = "H_M_CnstraintsL2"
VolumeIntegrals_vacuum::Integration_quantity_keyword[9] = "H_M_CnstraintsL2"
VolumeIntegrals_vacuum::Integration_quantity_keyword[10]= "H_M_CnstraintsL2"

# Second integral takes the first integral integrand,
# then excises the region around the first BH
VolumeIntegrals_vacuum::volintegral_sphere__center_x_initial            [2] = -5.9
VolumeIntegrals_vacuum::volintegral_outside_sphere__radius              [2] =  2.2
VolumeIntegrals_vacuum::volintegral_sphere__tracks__amr_centre          [2] =  0
VolumeIntegrals_vacuum::volintegral_usepreviousintegrands_num_integrands[2] =  4

# Third integral takes the second integral integrand,
# then excises the region around the first BH
VolumeIntegrals_vacuum::volintegral_sphere__center_x_initial            [3] =  5.9
VolumeIntegrals_vacuum::volintegral_outside_sphere__radius              [3] =  2.2
VolumeIntegrals_vacuum::volintegral_sphere__tracks__amr_centre          [3] =  1
VolumeIntegrals_vacuum::volintegral_usepreviousintegrands_num_integrands[3] =  4

# Just an outer region
VolumeIntegrals_vacuum::volintegral_outside_sphere__radius[4] = 8.2
VolumeIntegrals_vacuum::volintegral_outside_sphere__radius[5] =12.0
VolumeIntegrals_vacuum::volintegral_outside_sphere__radius[6] =16.0
VolumeIntegrals_vacuum::volintegral_outside_sphere__radius[7] =24.0
VolumeIntegrals_vacuum::volintegral_outside_sphere__radius[8] =48.0
VolumeIntegrals_vacuum::volintegral_outside_sphere__radius[9] =96.0
VolumeIntegrals_vacuum::volintegral_outside_sphere__radius[10]=192.0
\end{verbatim}
%%


%%%%%%%%%%%%%%%%%%%%%%%%%%
%%%%%%% References %%%%%%%
%%%%%%%%%%%%%%%%%%%%%%%%%%
\begin{thebibliography}{2}

\bibitem{alcubierre2008introduction}
  Alcubierre, M.
  \newblock {\em Introduction to 3+ 1 numerical relativity\/}, Vol 140. Oxford University Press (2008).

\bibitem{gourgoulhon20073+}
  Gourgoulhon, E.
  \newblock {\em 3+ 1 formalism and bases of numerical relativity\/}, arXiv preprint gr-qc/0703035 (2007).

\end{thebibliography}

% Do not delete next line
% END CACTUS THORNGUIDE

\end{document}
