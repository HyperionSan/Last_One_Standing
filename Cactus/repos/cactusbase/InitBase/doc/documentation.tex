% *======================================================================*
%  Cactus Thorn template for ThornGuide documentation
%  Author: Ian Kelley
%  Date: Sun Jun 02, 2002
%  $Header$
%
%  Thorn documentation in the latex file doc/documentation.tex
%  will be included in ThornGuides built with the Cactus make system.
%  The scripts employed by the make system automatically include
%  pages about variables, parameters and scheduling parsed from the
%  relevant thorn CCL files.
%
%  This template contains guidelines which help to assure that your
%  documentation will be correctly added to ThornGuides. More
%  information is available in the Cactus UsersGuide.
%
%  Guidelines:
%   - Do not change anything before the line
%       % START CACTUS THORNGUIDE",
%     except for filling in the title, author, date, etc. fields.
%        - Each of these fields should only be on ONE line.
%        - Author names should be separated with a \\ or a comma.
%   - You can define your own macros, but they must appear after
%     the START CACTUS THORNGUIDE line, and must not redefine standard
%     latex commands.
%   - To avoid name clashes with other thorns, 'labels', 'citations',
%     'references', and 'image' names should conform to the following
%     convention:
%       ARRANGEMENT_THORN_LABEL
%     For example, an image wave.eps in the arrangement CactusWave and
%     thorn WaveToyC should be renamed to CactusWave_WaveToyC_wave.eps
%   - Graphics should only be included using the graphicx package.
%     More specifically, with the "\includegraphics" command.  Do
%     not specify any graphic file extensions in your .tex file. This
%     will allow us to create a PDF version of the ThornGuide
%     via pdflatex.
%   - References should be included with the latex "\bibitem" command.
%   - Use \begin{abstract}...\end{abstract} instead of \abstract{...}
%   - Do not use \appendix, instead include any appendices you need as
%     standard sections.
%   - For the benefit of our Perl scripts, and for future extensions,
%     please use simple latex.
%
% *======================================================================*
%
% Example of including a graphic image:
%    \begin{figure}[ht]
% 	\begin{center}
%    	   \includegraphics[width=6cm]{MyArrangement_MyThorn_MyFigure}
% 	\end{center}
% 	\caption{Illustration of this and that}
% 	\label{MyArrangement_MyThorn_MyLabel}
%    \end{figure}
%
% Example of using a label:
%   \label{MyArrangement_MyThorn_MyLabel}
%
% Example of a citation:
%    \cite{MyArrangement_MyThorn_Author99}
%
% Example of including a reference
%   \bibitem{MyArrangement_MyThorn_Author99}
%   {J. Author, {\em The Title of the Book, Journal, or periodical}, 1 (1999),
%   1--16. {\tt http://www.nowhere.com/}}
%
% *======================================================================*

% If you are using CVS use this line to give version information
% $Header$

\documentclass{article}

% Use the Cactus ThornGuide style file
% (Automatically used from Cactus distribution, if you have a
%  thorn without the Cactus Flesh download this from the Cactus
%  homepage at www.cactuscode.org)
\usepackage{../../../../doc/latex/cactus}

\begin{document}

% The author of the documentation
\author{Erik Schnetter \textless schnetter@cct.lsu.edu\textgreater}

% The title of the document (not necessarily the name of the Thorn)
\title{InitBase}

% the date your document was last changed, if your document is in CVS,
% please use:
%    \date{$ $Date$ $}
\date{July 06 2007}

\maketitle

% Do not delete next line
% START CACTUS THORNGUIDE

% Add all definitions used in this documentation here
%   \def\mydef etc

% Add an abstract for this thorn's documentation
\begin{abstract}
  Thorn \texttt{InitBase} specifis how initial data are to be set up.
  It does not set up any initial data by itself, nor does it contain
  any routines which are to be called.  It is merely a convenient
  repository remembering how initial data are to be set up, so that
  other thorns can check their actions against this thorn.
  
  There are several possibilities:
  \begin{enumerate}
  \item The initial data thorn sets up data on one time level, while
    other time levels are scratch space.  The time evolution method
    must start up from a single time level.  (This is the default.)
  \item The initial data thorn sets up data on exactly one time level,
    and is called once for each active time level.  (This means that
    the initial data thorn can only access the current time level.)
  \item The initial data thorn sets up data on exactly two time
    levels, and is called once for each active time level.  (This
    means that the initial data thorn can only access the current and
    the first past time level.)
  \item The initial data thorn sets up data on all active time levels.
    (This makes it necessary that the initial data thorn checks the
    number of active time levels.)
  \end{enumerate}
  
  This thorn does not set up any initial data by itself.  It only
  selects the mechanism; it is the responsibility of other thorns to
  follow this specification.  If all other thorns follow this
  specification, then they are guaranteed to interact nicely.  If
  other thorns ignore this specification, then end-users may be
  surprised and may have difficulties setting up parameter files which
  work correctly.
\end{abstract}

% The following sections are suggestive only.
% Remove them or add your own.



%\begin{thebibliography}{9}
%
%\end{thebibliography}

% Do not delete next line
% END CACTUS THORNGUIDE

\end{document}
