\documentclass{article}

% Use the Cactus ThornGuide style file
% (Automatically used from Cactus distribution, if you have a 
%  thorn without the Cactus Flesh download this from the Cactus
%  homepage at www.cactuscode.org)
\usepackage{../../../../doc/latex/cactus}

\begin{document}

\title{Time}
\author{Gabrielle Allen}
\date{$ $Date$ $}

\maketitle

% Do not delete next line
% START CACTUS THORNGUIDE

\begin{abstract}
Calculates the timestep used for an evolution
\end{abstract}

\section{Purpose}

This thorn provides routines for calculating
the timestep for an evolution based on the spatial Cartesian grid spacing and
a wave speed. 

\section{Description}

Thorn {\tt Time} uses one of four methods to decide on the timestep
to be used for the simulation. The method is chosen using the
keyword parameter {\tt time::timestep\_method}. 

\begin{itemize}

\item{} {\tt time::timestep\_method = "given"} 

	The timestep is fixed to the
	value of the parameter {\tt time::timestep}. 

\item{} {\tt time::timestep\_method = "courant\_static"} 

	This is the default
	method, which calculates the timestep once at the start of the
	simulation, based on a simple courant type condition using 
	the spatial gridsizes and the parameter {\tt time::dtfac}.
$$
\Delta t = \mbox{\tt dtfac} * \mbox{min}(\Delta x^i)
$$
	Note that it is up to the user to custom {\tt dtfac} to take
	into account the dimension of the space being used, and the wave speed.

\item{} {\tt time::timestep\_method = "courant\_speed"} 

	This choice implements a 
	dynamic courant type condition, the timestep being set before each
	iteration using the spatial dimension of the grid, the spatial grid sizes, the 
	parameter {\tt courant\_fac} and the grid variable {\tt courant\_wave\_speed}. 
	The algorithm used is
$$
\Delta t = \mbox{\tt courant\_fac} * \mbox{min}(\Delta x^i)/\mbox{\tt courant\_wave\_speed}/\sqrt{\mbox dim}
$$
	For this algorithm to be successful, the variable {\tt courant\_wave\_speed}
	must have been set by some thorn to the maximum propagation speed on the grid {\it before}
	this thorn sets the timestep, that is {\tt AT POSTSTEP BEFORE Time\_Courant} (or earlier 
	in the evolution loop). [Note: The name {\tt courant\_wave\_speed} was poorly 
	chosen here, the required speed is the maximum propagation speed on 
        the grid which may be larger than the maximum wave speed (for example
        with a shock wave in hydrodynamics, also it is possible to have
        propagation without waves as with a pure advection equation).

\item{} {\tt time::timestep\_method = "courant\_time"} 

	This choice is similar to the
	method {\tt courant\_speed} above, in implementing a dynamic timestep.
	However the timestep is chosen using
$$
\Delta t = \mbox{\tt courant\_fac} * \mbox{\tt courant\_min\_time}/\sqrt{\mbox dim}
$$
        where the grid variable {\tt courant\_min\_time} must be set by some thorn to
	the minimum time for a wave to cross a gridzone {\it before}
	this thorn sets the timestep, that is {\tt AT POSTSTEP BEFORE Time\_Courant} (or earlier 
	in the evolution loop). 

\end{itemize}

In all cases, Thorn {\tt Time} sets the Cactus variable {\tt cctk\_delta\_time}
which is passed as part of the macro {\tt CCTK\_ARGUMENTS} to thorns called 
by the scheduler.

Note that for hyperbolic problems, the Courant condition gives a minimum 
requirement for stability, namely that the numerical domain of dependency
must encompass the physical domain of dependency, or
$$
\Delta t \le \mbox{min}(\Delta x^i)/\mbox{wave speed}/\sqrt{\mbox dim}
$$

\section{Examples}

\noindent
{\bf Fixed Value Timestep}

{\tt
\begin{verbatim}
time::timestep_method = "given"
time::timestep        = 0.1
\end{verbatim}
}


\noindent
{\bf Calculate Static Timestep Based on Grid Spacings}

\noindent
The following parameters set the timestep to be 0.25

{\tt
\begin{verbatim}
grid::dx    = 0.5
grid::dy    = 1.0
grid::dz    = 1.0
time::timestep_method = "courant_static"
time::dtfac = 0.5
\end{verbatim}
}

% Do not delete next line
% END CACTUS THORNGUIDE

\end{document}
