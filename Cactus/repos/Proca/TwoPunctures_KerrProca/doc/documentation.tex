% *======================================================================*
%  Cactus Thorn template for ThornGuide documentation
%  Author: Ian Kelley
%  Date: Sun Jun 02, 2002
%
%  Thorn documentation in the latex file doc/documentation.tex
%  will be included in ThornGuides built with the Cactus make system.
%  The scripts employed by the make system automatically include
%  pages about variables, parameters and scheduling parsed from the
%  relevant thorn CCL files.
%
%  This template contains guidelines which help to assure that your
%  documentation will be correctly added to ThornGuides. More
%  information is available in the Cactus UsersGuide.
%
%  Guidelines:
%   - Do not change anything before the line
%       % START CACTUS THORNGUIDE",
%     except for filling in the title, author, date, etc. fields.
%        - Each of these fields should only be on ONE line.
%        - Author names should be separated with a \\ or a comma.
%   - You can define your own macros, but they must appear after
%     the START CACTUS THORNGUIDE line, and must not redefine standard
%     latex commands.
%   - To avoid name clashes with other thorns, 'labels', 'citations',
%     'references', and 'image' names should conform to the following
%     convention:
%       ARRANGEMENT_THORN_LABEL
%     For example, an image wave.eps in the arrangement CactusWave and
%     thorn WaveToyC should be renamed to CactusWave_WaveToyC_wave.eps
%   - Graphics should only be included using the graphicx package.
%     More specifically, with the "\includegraphics" command.  Do
%     not specify any graphic file extensions in your .tex file. This
%     will allow us to create a PDF version of the ThornGuide
%     via pdflatex.
%   - References should be included with the latex "\bibitem" command.
%   - Use \begin{abstract}...\end{abstract} instead of \abstract{...}
%   - Do not use \appendix, instead include any appendices you need as
%     standard sections.
%   - For the benefit of our Perl scripts, and for future extensions,
%     please use simple latex.
%
% *======================================================================*
%
% Example of including a graphic image:
%    \begin{figure}[ht]
% 	\begin{center}
%    	   \includegraphics[width=6cm]{MyArrangement_MyThorn_MyFigure}
% 	\end{center}
% 	\caption{Illustration of this and that}
% 	\label{MyArrangement_MyThorn_MyLabel}
%    \end{figure}
%
% Example of using a label:
%   \label{MyArrangement_MyThorn_MyLabel}
%
% Example of a citation:
%    \cite{MyArrangement_MyThorn_Author99}
%
% Example of including a reference
%   \bibitem{MyArrangement_MyThorn_Author99}
%   {J. Author, {\em The Title of the Book, Journal, or periodical}, 1 (1999),
%   1--16. {\tt http://www.nowhere.com/}}
%
% *======================================================================*

\documentclass{article}

% Use the Cactus ThornGuide style file
% (Automatically used from Cactus distribution, if you have a
%  thorn without the Cactus Flesh download this from the Cactus
%  homepage at www.cactuscode.org)
\usepackage{../../../../doc/latex/cactus}

\begin{document}

% The author of the documentation
\author{Helvi Witek, Miguel Zilh\~ao}

% The title of the document (not necessarily the name of the Thorn)
\title{TwoPunctures\_KerrProca}

% the date your document was last changed:
\date{October 26 2018}

\maketitle

% Do not delete next line
% START CACTUS THORNGUIDE

% Add all definitions used in this documentation here
%   \def\mydef etc

% Add an abstract for this thorn's documentation
% \begin{abstract}

% \end{abstract}

% The following sections are suggestive only.
% Remove them or add your own.

\section{Notes}

We modify the \texttt{TwoPunctures} thorn to generate initial data for a single rotating
black hole coupled to a massive vector field.  The conformal metric is related
to Kerr in quasi-isotropic coordinates (see~\cite{Liu:2009al} and
\cite{Okawa:2014nda})), i.e., $rBL = R ( 1 + rBL_{+} / (4R) )^{2} $. This construction can accommodate for Proca fields with the following initial angular distribution :$A_{\phi} \sim Y_{00}$, $A_{\phi} \sim Y_{10}$ (explored in \cite{Zilhao:2015tya}) and $A_{\phi} \sim Y_{1-1} - Y_{11}$ (see \cite{Witek:2018tba}).

\begin{description}

\item[Note:] We are enabling the swapping parameter ``swap\_xz'', i.e., the x- and
  z-axis are simply exchanged because \texttt{TwoPunctures} uses coordinates such that
  $x=R\cos\theta$.  Therefore, the rotation axis is the x-axis, and one needs to
  specify \texttt{par\_S\_plus[0]}.

  The swapping is \emph{not} a coordinate transformation!  So, instead of
  going from $x=R\cos\theta, y=R\sin\theta\cos\phi, z=R\sin\theta\sin\phi$ to
  $x=R\sin\theta\cos\phi, y=R\sin\theta\sin\phi, z=R\cos\theta$ as one might
  expect, this operation only exchanges x-z.  Hence, for a BH that in physical
  space is supposed to rotate along $z$, \emph{in the direction of $z$}, needs
  to be specified with a minus sign.

  As an example, say we want to set up a rotating BH with $a/M = 0.9$, rotating
  in the z-direction. Then, we need
    \begin{itemize}
    \item \texttt{swap\_xz = yes}
    \item \texttt{par\_b = 1.0}
    \item \texttt{offset[2]= -1.0}
    \item \texttt{par\_S\_plus[0] = -0.90}
    \end{itemize}

\end{description}






\begin{thebibliography}{9}

%\cite{Liu:2009al}
\bibitem{Liu:2009al}
  Y.~T.~Liu, Z.~B.~Etienne and S.~L.~Shapiro,
  ``Evolution of near-extremal-spin black holes using the moving puncture technique,''
  Phys.\ Rev.\ D {\bf 80} (2009) 121503
  doi:10.1103/PhysRevD.80.121503
  [arXiv:1001.4077 [gr-qc]].
  %%CITATION = doi:10.1103/PhysRevD.80.121503;%%
  %18 citations counted in INSPIRE as of 26 Oct 2018

%\cite{Okawa:2014nda}
\bibitem{Okawa:2014nda}
  H.~Okawa, H.~Witek and V.~Cardoso,
  ``Black holes and fundamental fields in Numerical Relativity: initial data construction and evolution of bound states,''
  Phys.\ Rev.\ D {\bf 89} (2014) no.10,  104032
  doi:10.1103/PhysRevD.89.104032
  [arXiv:1401.1548 [gr-qc]].
  %%CITATION = doi:10.1103/PhysRevD.89.104032;%%
  %58 citations counted in INSPIRE as of 26 Oct 2018

% \cite{Zilhao:2015tya}
\bibitem{Zilhao:2015tya}
  M.~Zilh\~ao, H.~Witek and V.~Cardoso,
  ``Nonlinear interactions between black holes and Proca fields,''
  Class.\ Quant.\ Grav.\  {\bf 32} (2015) 234003
  doi:10.1088/0264-9381/32/23/234003
  [arXiv:1505.00797 [gr-qc]].
  %%CITATION = doi:10.1088/0264-9381/32/23/234003;%%
  %25 citations counted in INSPIRE as of 26 Oct 2018

\bibitem{Witek:2018tba}
  M.~Zilh\~ao, H.~Witek. To appear.


\end{thebibliography}

% Do not delete next line
% END CACTUS THORNGUIDE

\end{document}
