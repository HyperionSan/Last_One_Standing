% If you are using CVS use this line to give version information
% $Header$

\documentclass{article}

% Use the Cactus ThornGuide style file
% (Automatically used from Cactus distribution, if you have a 
%  thorn without the Cactus Flesh download this from the Cactus
%  homepage at www.cactuscode.org)
\usepackage{../../../../doc/latex/cactus}

\begin{document}

\title{IDScalarWave}
\author{Gabrielle Allen \\ Horst Beyer}
\date{$ $Date$ $}

\maketitle

% Do not delete next line
% START CACTUS THORNGUIDE

\begin{abstract}
Initial Data for the 3D Scalar Wave Equation
\end{abstract}

\section{Purpose}

This thorn provides different initial data for the 3D scalar wave
equation. 

\section{Spherically Symmetric Solutions}

The general spherically symmetric solution can be written
\begin{equation}
\Psi(r,t) = \frac{1}{r}\left(f(r+t)+g(r-t)\right)
\end{equation}
where the functions $f$ and $g$ can be freely chosen. 

Making the additional requirement of time symmetry at $t=0$, forces
\begin{equation}
f(r)=g(r)
\end{equation}
Thus if the solution at t=0 is given by $\phi(r)$, the general solution
will be
\begin{equation}
\Psi(r,t) = \frac{1}{2r}\left( (r+t)\phi(r+t)+(r-t)\phi(r-t) \right)
\end{equation}

\section{Gaussian}

The gaussian solution is {\it spherically symmetric} about the 
origin of the Cartesian coordinate system, and is {\it time symmetric}.
The initial profile is
\begin{equation}
\phi(r) = A \exp (- r^2/\sigma)
\end{equation}
with the solution at the origin being
\begin{equation}
\Psi(r=0,t) = \left(1-2\frac{t^2}{\sigma}\right)\exp(-t^2/\sigma)
\end{equation}

The Gaussian solution is set with the parameters
\begin{itemize}

\item {\tt amplitude} = $A$

\item {\tt sigma} = $\sigma$

\end{itemize}


% Do not delete next line
% END CACTUS THORNGUIDE

\end{document}
