\documentclass{article}
\usepackage{../../../doc/latex/cactus}

\begin{document}

\title{CactusWave}
\author{Gabrielle Allen, Tom Goodale}
\date{1999}
\maketitle

\abstract{Set of thorns for evolving the standard 3D scalar wave equation. The
package includes initial data, evolver, and analysis thorns in a variety of
programming languages.}

\section{Purpose}

To demonstrate the use of the Cactus code through a simple, illustrative
example.

The model problem solved is the 3D scalar wave equation in 
Cartesian coordinates,
$$
\frac{\partial^2 \phi}{\partial t^2} =
  \frac{\partial^2 \phi}{\partial x^2} +
  \frac{\partial^2 \phi}{\partial y^2} +
  \frac{\partial^2 \phi}{\partial z^2} 
$$
The numerical solution of this equation requires initial data
to be specified for
$$
\phi(t=0), \qquad \frac{\partial \phi}{\partial t}(t=0)
$$
The numerical method employed in these thorns to solve for $\phi$
is a standard 2nd order centered finite difference method.
The solution $\phi(t,x,y,z)$ is discretised using
$$
  \phi(t_i,x_i,y_i,z_i) = \phi^n_{i,j,k}
$$
where, for example,
$$
x_i = x_0 + i \Delta x
$$
The solution at any timeslice can then be found iteratively 
using the previous two timeslices using the algorithm
\begin{eqnarray}
\phi^{n+1} &=& 
2(1- \rho_x^2 - \rho_y^2 - \rho_z^2) \phi^n_{i,j,k}
-\phi^{n-1}_{i,j,k} 
+ \rho_x^2(\phi^n_{i+1,j,k}-\phi^n_{i-1,j,k})
\nonumber\\
&&+ \rho_y^2(\phi^n_{i,j+1,k}-\phi^n_{i,j-1,k})
+ \rho_z^2(\phi^n_{i,j,k+1}-\phi^n_{i,j,k-1})
\end{eqnarray}
where we define the Courant factors
$$
\rho_x = \frac{\Delta t}{\Delta x} \qquad
\rho_y = \frac{\Delta t}{\Delta y} \qquad
\rho_z = \frac{\Delta t}{\Delta z}
$$

\section{Comments}

Here we give a brief description of each of the thorns contained in
this arrangement

\begin{itemize}

\item{\bf IDScalarWave} Different initial data sets, all of which 
   are analytic. 

\item{\bf IDScalarWaveCXX} The same as IDScalarWave but implemented in C++. 

\item{\bf IDScalarWaveElliptic} Initial data sets from solving an 
   elliptic equation. At the moment this initial data is rather 
   artificial, and is just here to give a simple demonstration of
   using an elliptic solver.

\item{\bf WaveToyC} The evolver for the scalar field, written in C.

\item{\bf WaveToyF77} The same as WaveToyC, but written in F77 to 
   demonstrate the use of {\tt implementation}s.

\item{\bf WaveToyF90} The same as the two evolver thorns above, 
   but this time to show the difference between F77 and F90, and
   to further demonstrate {\tt implementation}s.

\item{\bf WaveToyFreeF90} The same as WaveToyF90, but written with 
   free-format F90 rather than fixed.

\item{\bf WaveToyCXX} The same as WaveToyC, but written in C++.

%\item{\bf WaveAnalysis} What is this?

\end{itemize}

\end{document}
