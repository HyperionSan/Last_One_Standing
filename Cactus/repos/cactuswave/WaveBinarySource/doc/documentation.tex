% If you are using CVS use this line to give version information
% $Header$

\documentclass{article}

% Use the Cactus ThornGuide style file
% (Automatically used from Cactus distribution, if you have a 
%  thorn without the Cactus Flesh download this from the Cactus
%  homepage at www.cactuscode.org)
\usepackage{../../../../doc/latex/cactus}

\begin{document}

% The author of the documentation
\author{Gabrielle Allen} 

% The title of the document (not necessarily the name of the Thorn)
\title{Orbiting Scalar Charges}

% the date your document was last changed, if your document is in CVS, 
% please us:
\date{$ $Date$ $}


\maketitle

% Do not delete next line
% START CACTUS THORNGUIDE

% Add all definitions used in this documentation here 
%   \def\mydef etc

% Add an abstract for this thorn's documentation
\begin{abstract}

\end{abstract}

% The following sections are suggestive only.
% Remove them or add your own.

\section{Introduction}

This thorn provides a source term to the scalar field evolution
for two rotating binary charges.


\section{Physical System}

The 3D scalar wave equation with a source term $\rho(t,x,y,z)$ is written
$$
\nabla \phi = 4 \pi \rho
$$

Each scalar source with charge $Q$ and radius $R$ contributes 
$$
\rho = \frac{3Q}{4\pi R^3}
$$

\section{Numerical Implementation}

The only involved part of this thorn arise in working out where the
sources are located (if at all) on each local grid for a multiprocessor
run. The source terms are not numerically evolved, but are calculated 
exactly, based on the physical time and their orbital velocity.

A routine is scheduled to run {\it after} the homogeneous equation
for the scalar field has been evolved, and simply updates the value
of the scalar field by adding on the source contribution.

% Do not delete next line
% END CACTUS THORNGUIDE

\end{document}
