% *======================================================================*
%  Cactus Thorn template for ThornGuide documentation
%  Author: Ian Kelley
%  Date: Sun Jun 02, 2002
%  $Header$                                                             
%
%  Thorn documentation in the latex file doc/documentation.tex 
%  will be included in ThornGuides built with the Cactus make system.
%  The scripts employed by the make system automatically include 
%  pages about variables, parameters and scheduling parsed from the 
%  relevent thorn CCL files.
%  
%  This template contains guidelines which help to assure that your     
%  documentation will be correctly added to ThornGuides. More 
%  information is available in the Cactus UsersGuide.
%                                                    
%  Guidelines:
%   - Do not change anything before the line
%       % START CACTUS THORNGUIDE",
%     except for filling in the title, author, date etc. fields.
%        - Each of these fields should only be on ONE line.
%        - Author names should be sparated with a \\ or a comma
%   - You can define your own macros, but they must appear after
%     the START CACTUS THORNGUIDE line, and must not redefine standard 
%     latex commands.
%   - To avoid name clashes with other thorns, 'labels', 'citations', 
%     'references', and 'image' names should conform to the following 
%     convention:          
%       ARRANGEMENT_THORN_LABEL
%     For example, an image wave.eps in the arrangement CactusWave and 
%     thorn WaveToyC should be renamed to CactusWave_WaveToyC_wave.eps
%   - Graphics should only be included using the graphix package. 
%     More specifically, with the "includegraphics" command. Do
%     not specify any graphic file extensions in your .tex file. This 
%     will allow us (later) to create a PDF version of the ThornGuide
%     via pdflatex. |
%   - References should be included with the latex "bibitem" command. 
%   - Use \begin{abstract}...\end{abstract} instead of \abstract{...}
%   - Do not use \appendix, instead include any appendices you need as 
%     standard sections. 
%   - For the benefit of our Perl scripts, and for future extensions, 
%     please use simple latex.     
%
% *======================================================================* 
% 
% Example of including a graphic image:
%    \begin{figure}[ht]
%       \begin{center}
%          \includegraphics[width=6cm]{MyArrangement_MyThorn_MyFigure}
%       \end{center}
%       \caption{Illustration of this and that}
%       \label{MyArrangement_MyThorn_MyLabel}
%    \end{figure}
%
% Example of using a label:
%   \label{MyArrangement_MyThorn_MyLabel}
%
% Example of a citation:
%    \cite{MyArrangement_MyThorn_Author99}
%
% Example of including a reference
%   \bibitem{MyArrangement_MyThorn_Author99}
%   {J. Author, {\em The Title of the Book, Journal, or periodical}, 1 (1999), 
%   1--16. {\tt http://www.nowhere.com/}}
%
% *======================================================================* 

% If you are using CVS use this line to give version information
% $Header$

\documentclass{article}

% Use the Cactus ThornGuide style file
% (Automatically used from Cactus distribution, if you have a 
%  thorn without the Cactus Flesh download this from the Cactus
%  homepage at www.cactuscode.org)
\usepackage{../../../../doc/latex/cactus}

\begin{document}

% The author of the documentation
\author{Ian Hawke} 

% The title of the document (not necessarily the name of the Thorn)
\title{The scalar wave equation in Method of Lines form}

% the date your document was last changed, if your document is in CVS, 
% please use:
\date{$ $Date$ $}

\maketitle

% Do not delete next line
% START CACTUS THORNGUIDE

% Add all definitions used in this documentation here 
%   \def\mydef etc

% Add an abstract for this thorn's documentation
\begin{abstract}
  WaveMoL is an example implementation of a thorn that uses the method
  of lines thorn MoL. The wave equation in first order form is
  implemented. 
\end{abstract}

% The following sections are suggestive only.
% Remove them or add your own.

% \section{Introduction}

% \section{Physical System}

% \section{Numerical Implementation}

% \section{Using This Thorn}

% \subsection{Obtaining This Thorn}

% \subsection{Basic Usage}

% \subsection{Special Behaviour}

% \subsection{Interaction With Other Thorns}

% \subsection{Examples}

% \subsection{Support and Feedback}

% \section{History}

% \subsection{Thorn Source Code}

% \subsection{Thorn Documentation}

% \subsection{Acknowledgements}

\section{Purpose}
\label{sec:purpose}

WaveToy is the simple test thorn that comes with Cactus as
standard. This is written so that it solves the wave equation
\begin{equation}
  \label{eq:wave1}
  \partial_t^2 \phi = \partial_{x^i}^2 \phi^i
\end{equation}
directly using the leapfrog scheme. This form of the equations isn't
suitable for use with the method of lines.

The purpose of this thorn is to rewrite the equations in first order
form
\begin{eqnarray}
  \label{eq:wave2}
  \partial_t \Phi & = & \partial_{x^i} \Pi^i, \\
  \partial_t \Pi^j & = & \partial_{x^j} \Phi, \\
  \partial_t \phi & = & \Phi, \\
  \partial_{x^j} \phi & = & \Pi^j.
\end{eqnarray}
The first three equations (which expand to five separate PDEs) will be
evolved. The final equation is used to set the initial data and can be
thought of as a constraint.

This will be implemented using simple second order differencing in
space. Time evolution is performed by the method of lines thorn MoL.

\section{How it works}
\label{sec:details}

The equations are evolved entirely using the method of lines thorn. So
all we have to provide (for the evolution) is a method of calculating
the right hand side of equation~(\ref{eq:wave2}) and boundary
conditions. The boundary conditions are standard from wavetoy
itself. The right hand side is calculated using second order centred
finite differences.

To be compatible with the method of lines thorn we must let it know
that the GFs $(\phi, \Pi, \Phi^j)$ exist and where they are stored.
They should all have at least two time levels (although the addition
of extra time levels may not cause problems it's just wasting space).
The GFs corresponding to the right hand sides must also be registered
with the MoL thorn. These should only have one time level. MoL is
informed of the existence of these grid functions through the
accumulator parameters and the aliased functions.

% Do not delete next line
% END CACTUS THORNGUIDE

\end{document}
