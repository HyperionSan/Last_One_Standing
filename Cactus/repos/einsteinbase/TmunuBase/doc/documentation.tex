%  Thorn documentation in the latex file doc/documentation.tex
%  will be included in ThornGuides built with the Cactus make system.
%  The scripts employed by the make system automatically include
%  pages about variables, parameters and scheduling parsed from the
%  relevant thorn CCL files.

\documentclass{article}

% Use the Cactus ThornGuide style file
% (Automatically used from Cactus distribution, if you have a
%  thorn without the Cactus Flesh download this from the Cactus
%  homepage at www.cactuscode.org)
\usepackage{../../../../doc/latex/cactus}

\begin{document}

\author{Erik Schnetter \textless schnetter@cct.lsu.edu\textgreater}

\title{TmunuBase}

\date{2008-04-07}

\maketitle

% Do not delete next line
% START CACTUS THORNGUIDE

\begin{abstract}
  Provide grid functions for the stress-energy tensor $T_{\mu\nu}$,
  and schedule when these grid functions are calculated.  This allows
  different thorns to cooperate without explicitly depending on each
  other.  Thorn TmunuBase is for the stress-energy
  tensor what thorn ADMBase is for the metric tensor.
\end{abstract}

\section{Introduction}

Thorn \texttt{TmunuBase} provides core infrastructure for thorns
implementing some kind of energy or matter in general relativity, for
example general relativistic hydrodynamics formulations.  It provides
the basic variables, i.e., the stress-energy tensor $T_{\mu\nu}$, in
addition to a set of parameters to regulate their use.  These
variables are used to communicate between (possibly multiple) thorns
contributing to the stress-energy content of the spacetime, and thorns
needing to evaluate the stress-energy tensor such as spacetime
evolution methods.  It also provides schedule groups to manage when
$T_{\mu\nu}$ is calculated and when it is ready for access.

\section{Using TmunuBase}

\subsection{Variables}

TmunuBase weakly assumes (but does not require) that the spacetime is
described in terms of a $3+1$ decomposition.  The variables provided
by \texttt{TmunuBase} are:
\begin{itemize}
\item The ``scalar'' part of $T_{\mu\nu}$, its time-time component:
  \texttt{eTtt}
\item The ``vector'' part of $T_{\mu\nu}$, its time-space components:
  \texttt{eTtx}, \texttt{eTty}, \texttt{eTtz}
\item The ``tensor'' part of $T_{\mu\nu}$, its space-space components:
  \texttt{eTxx}, \texttt{eTxy}, \texttt{eTxz}, \texttt{eTyy},
  \texttt{eTyz}, \texttt{eTzz}
\end{itemize}
These components have the prefix \texttt{e} to avoid naming conflicts
with existing variables.  Many thorns dealing with matter already use
variable names such as \texttt{Ttt}.

These variables have up to three time levels.

\subsection{Parameters}

By default, the TmunuBase variables have no storage, and TmunuBase is
inactive.  This makes it possible to add a matter interface to
existing vacuum spacetime methods without changing their behaviour.

Several parameters choose how TmunuBase behaves at run time:
\begin{itemize}
\item The parameter \texttt{stress\_energy\_storage} activates storage
  for $T_{\mu\nu}$ and enables the schedule groups which calculate it.
\item The parameter \texttt{stress\_energy\_at\_RHS} moves calculating
  the $T_{\mu\nu}$ from the \texttt{evol} bin into the
  \texttt{MoL\_PostStep} group.  This increases the order of accuracy
  of the spacetime--matter coupling, but is only possible when thorn
  MoL is used.\footnote{This was one of the main reason why thorn MoL
    was instroduced.}  Generally, this parameter should be set when
  MoL is used.
\item The parameter \texttt{timelevels} chooses the number of time
  levels for $T_{\mu\nu}$.  The default is a single time level, which
  is sufficient for unigrid simulation.  Mesh refinement simulation
  may require several time levels if mesh refinement boundaries
  require correct values.
\item The parameter \texttt{prolongation\_type} defines the
  prolongation operator for mesh refinement boundaries.  The default
  is Lagrange interpolation.
\end{itemize}

The grid scalar \texttt{stress\_energy\_state} describes whether the
$T_{\mu\nu}$ variables have storage.

\section{Programming with TmunuBase}

\subsection{Contributing to $T_{\mu\nu}$}

There may be multiple thorns contributing to $T_{\mu\nu}$.  Therefore,
thorn TmunuBase initialises $T_{\mu\nu}$ to zero, and each thorn has
to add to the existing values in $T_{\mu\nu}$.  The corresponding
routine should be scheduled in the bin \texttt{AddToTmunu}.
\emph{Note:} Do not schedule anything in the schedule bin
\texttt{SetTmunu}.

\subsection{Reading from $T_{\mu\nu}$}

Since the values of $T_{\mu\nu}$ change at each time step, or -- if a
thorn like \texttt{MoL} is used -- at each substep, $T_{\mu\nu}$ needs
to be recalculated frequently.  This happens either in the schedule
bin \texttt{evol} or in the schedule group \texttt{MoL\_PostStep}.
$T_{\mu\nu}$ may only be accessed after it has been calculated, e.g.\
\texttt{IN MoL\_PostStep AFTER SetTmunu}.  $T_{\mu\nu}$ can be freely
accessed at other times, e.g.\ in \texttt{MoL\_CalcRHS} or at
\texttt{poststep} or \texttt{analyisis}.

\subsection{Acknowledgements}

We thank I. Hawke for designing and implementing thorn MoL, without
which a generic high-order coupling between spacetime and
hydrodynamics methods would not be possible.

% Do not delete next line
% END CACTUS THORNGUIDE

\end{document}
