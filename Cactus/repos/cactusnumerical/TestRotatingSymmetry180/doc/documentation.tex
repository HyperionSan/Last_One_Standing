% *======================================================================*
%  Cactus Thorn template for ThornGuide documentation
%  Author: Ian Kelley
%  Date: Sun Jun 02, 2002
%  $Header$
%
%  Thorn documentation in the latex file doc/documentation.tex
%  will be included in ThornGuides built with the Cactus make system.
%  The scripts employed by the make system automatically include
%  pages about variables, parameters and scheduling parsed from the
%  relevant thorn CCL files.
%
%  This template contains guidelines which help to assure that your
%  documentation will be correctly added to ThornGuides. More
%  information is available in the Cactus UsersGuide.
%
%  Guidelines:
%   - Do not change anything before the line
%       % START CACTUS THORNGUIDE",
%     except for filling in the title, author, date, etc. fields.
%        - Each of these fields should only be on ONE line.
%        - Author names should be separated with a \\ or a comma.
%   - You can define your own macros, but they must appear after
%     the START CACTUS THORNGUIDE line, and must not redefine standard
%     latex commands.
%   - To avoid name clashes with other thorns, 'labels', 'citations',
%     'references', and 'image' names should conform to the following
%     convention:
%       ARRANGEMENT_THORN_LABEL
%     For example, an image wave.eps in the arrangement CactusWave and
%     thorn WaveToyC should be renamed to CactusWave_WaveToyC_wave.eps
%   - Graphics should only be included using the graphicx package.
%     More specifically, with the "\includegraphics" command.  Do
%     not specify any graphic file extensions in your .tex file. This
%     will allow us to create a PDF version of the ThornGuide
%     via pdflatex.
%   - References should be included with the latex "\bibitem" command.
%   - Use \begin{abstract}...\end{abstract} instead of \abstract{...}
%   - Do not use \appendix, instead include any appendices you need as
%     standard sections.
%   - For the benefit of our Perl scripts, and for future extensions,
%     please use simple latex.
%
% *======================================================================*
%
% Example of including a graphic image:
%    \begin{figure}[ht]
% 	\begin{center}
%    	   \includegraphics[width=6cm]{MyArrangement_MyThorn_MyFigure}
% 	\end{center}
% 	\caption{Illustration of this and that}
% 	\label{MyArrangement_MyThorn_MyLabel}
%    \end{figure}
%
% Example of using a label:
%   \label{MyArrangement_MyThorn_MyLabel}
%
% Example of a citation:
%    \cite{MyArrangement_MyThorn_Author99}
%
% Example of including a reference
%   \bibitem{MyArrangement_MyThorn_Author99}
%   {J. Author, {\em The Title of the Book, Journal, or periodical}, 1 (1999),
%   1--16. {\tt http://www.nowhere.com/}}
%
% *======================================================================*

% If you are using CVS use this line to give version information
% $Header$

\documentclass{article}

% Use the Cactus ThornGuide style file
% (Automatically used from Cactus distribution, if you have a
%  thorn without the Cactus Flesh download this from the Cactus
%  homepage at www.cactuscode.org)
\usepackage{../../../../doc/latex/cactus}

\begin{document}

% The author of the documentation
\author{Roland Haas \textless rhaas@illinois.edu\textgreater}

% The title of the document (not necessarily the name of the Thorn)
\title{TestRotatingSymmetry180}

% the date your document was last changed, if your document is in CVS,
% please use:
%    \date{$ $Date$ $}
\date{April 26 2019}

\maketitle

% Do not delete next line
% START CACTUS THORNGUIDE

% Add all definitions used in this documentation here
%   \def\mydef etc

% Add an abstract for this thorn's documentation
\begin{abstract}

This thorn tests RotatingSymmetry180 by setting up grid functons with known
symmetries and testing the RotatingSymmety180 applies the correct
transformation.

\end{abstract}

% The following sections are suggestive only.
% Remove them or add your own.

\section{Introduction}

RotatingSymmetry180 handles the symmetry transformation of tensorial
quantities under a transformation $(x,y,z) \rightarrow (-x,-y,z)$. The
tensorial nature of groups of grid functions is described via
\code{tensortypealias} tags to the grid functions.

This thorn exercises these capabilities by setting up grid functions of each
tensortype both single ones and vector grid functions.

\section{Physical System}

This thorns uses an expression of the form
\begin{equation}
T_{\mu\nu\lambda\ldots}(x,y,z) \propto x^\mu x^\nu x^\lambda \ldots
\end{equation}
where $x^\mu = {r, x, y, z}$ for $\mu = {0,1,2,3}$ i.e. it substitutes the
radius for the usual $t$ coordinate slot. This expression gives the
prototypical tensorial transformation behaviour under spatial transformations
for positive $+1$ parity (i.e. scalars and tensors but not psedudoscalars and
pseudotensors).

In addition explicit transformation rules for the Weyl scalars are included.

\section{Numerical Implementation}

The thorn first sets up grid functions with the correct values everywhere but
in symmetry boundaries, then applies the boundary condition, then counts how
many grid points (anywhere) have the wrong value. This difference is stored in
the grid scalar \code{num\_diffs}.

\section{Using This Thorn}

The thorn is intended to be used only as a testsuite unit test and comes with
a test \code{rotating180.par} that shows how to use it.

\subsection{Obtaining This Thorn}

\subsection{Special Behaviour}

This thorn currently hard-codes the boundary with and ghost zone with to 1.
The current implementation uses the names of the grid functions (more
specifically the part after the last underscore ``\verb|_|'') to extract the
tensor index of a given grid function.

\subsection{Interaction With Other Thorns}

This thorn requires \code{RotatingSymmetry180}.

\subsection{Support and Feedback}

Please contact the author(s) or maintainer(s) via eamil.

\section{History}

TODO: add support for \code{tensorparity=-1}.

% Do not delete next line
% END CACTUS THORNGUIDE

\end{document}
