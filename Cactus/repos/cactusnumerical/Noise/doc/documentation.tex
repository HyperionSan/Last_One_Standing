% /*@@
%   @file      documentation.tex
%   @date      16 May 2002
%   @author    Denis Pollney
%   @desc 
%              Thorn Noise user's guide.
%   @enddesc 
%   @version $Header$
% @@*/

\documentclass{article}

\usepackage{../../../../doc/latex/cactus}

\begin{document}
\title{Using the \texttt{Noise} thorn}
\author{Denis Pollney}
\date{April 2002}

\maketitle

% Do not delete next line
% START CACTUS THORNGUIDE

\begin{abstract}
The \texttt{Noise} thorn can be used to place random values on
Cactus grid functions at initial data and at the boundaries during
evolution. This can be used to carry out ``robust stability'' tests,
such as those proposed by Jeff Winicour.\\
\end{abstract}

\section{Initial data}

To apply a random perturbation to initial data, set
\texttt{noise::apply\_id\_noise="yes"}. Then each grid function
listed in the parameter \texttt{noise::id\_vars} will be adjusted
by a random factor. The maximum size of the random perturbation is
controlled by the parameter \texttt{noise::amplitude}. The
perturbations are applied during the \texttt{CCTK\_POSTINITIAL}
time bin.

\section{Boundary conditions}

A random number will be added to each point on the boundary of grid
functions listed in the \texttt{noise::bc\_vars} parameter if the
flag \texttt{noise::apply\_bc\_noise="yes"} is set. As with the
initial data, the maximum size of the perturbation is given by the
\texttt{noise::amplitude} parameter. The adjustments are applied at
each \texttt{CCTK\_POSTSTEP}.

\section{Example}

The following parameters can be used to apply a random adjustment 
of size $A=\pm0.0005$ to the initial data and boundaries of the metric
variables.
\begin{verbatim}
  ActiveThorns = "... Noise ..."

  noise::apply_id_noise = "yes"
  noise::id_vars        = "admbase::metric"

  noise::apply_bc_noise = "yes"
  noise::bc_vars        = "admbase::metric"

  noise::amplitude      = 0.001
\end{verbatim}

% Do not delete next line
% END CACTUS THORNGUIDE

\end{document}
