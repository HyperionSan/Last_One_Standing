\documentclass{article}

\begin{document}

\title{EOS\_Polytrope}
\author{Ian Hawke}
\date{22/4/2002}
\maketitle

% Do not delete next line
% START CACTUS THORNGUIDE

\abstract{EOS\_Polytrope} 

\section{The equations}
\label{sec:eqn}

This equation provides a polytropic equation of state to thorns using
the CactusEOS interface found in EOS\_Base. As such it's a fake, as
EOS\_Base assumes that, e.g., the pressure is a function of both
density and specific internal energy. Here the pressure is just a
function of the density, and is set appropriately (the specific
internal energy is always ignored).

The two fluid constants are $K$ ({\tt eos\_k}) and $\Gamma$ ({\tt
  eos\_gamma}), which default to 100 and 2 respectively. The formulas
that are applied under the appropriate EOS\_Base function calls are

\begin{eqnarray}
  \label{eq:eosformulas}
  P & = & K \rho^{\Gamma} \\
  \epsilon & = & \frac{K \rho^{\Gamma-1}}{\Gamma - 1} \\
  \rho & = & \frac{P}{(\Gamma - 1) \epsilon} \\
  \frac{\partial P}{\partial \rho} & = & K \Gamma \rho^{\Gamma-1} \\
  \frac{\partial P}{\partial \epsilon} & = & 0.
\end{eqnarray}

To calculate the units of the Cactus quantities and back, remember that
$G=c=M_{\odot}=1$ in Cactus.\\
Here is one example how to calculate densities:
\begin{equation}
 \rho_{\mbox{\tiny Cactus}}=\frac{G^3M_{\odot}^2}{c^6}\cdot \rho
 \approx1.6167\cdot10^{-21}\frac{\mbox{m}^3}{\mbox{kg}}\cdot\rho=
        1.6167\cdot10^{-18}\frac{\mbox{cm}^3}{\mbox{g}}\cdot\rho
\end{equation}
and one example for calculating $K$ (for $\Gamma=2$):
\begin{equation}
 K_{\mbox{\tiny Cactus}}=\frac{c^4}{G^3M_{\odot}^2}\cdot K
 \approx6.8824\cdot10^{-11}\frac{\mbox{m}^5}{\mbox{kg}\cdot\mbox{s}^2}\cdot K=
        6.8824\cdot10^{-4}\frac{\mbox{cm}^5}{\mbox{g}\cdot\mbox{s}^2}\cdot K
\end{equation}

% Do not delete next line
% END CACTUS THORNGUIDE

\end{document}
