\documentclass{article}

% Use the Cactus ThornGuide style file
% (Automatically used from Cactus distribution, if you have a 
%  thorn without the Cactus Flesh download this from the Cactus
%  homepage at www.cactuscode.org)
\usepackage{../../../../doc/latex/cactus}

\begin{document}

\title{ADMAnalysis}
\author{Tom Goodale et al}
\date{$ $Date$ $}

\maketitle

% Do not delete next line
% START CACTUS THORNGUIDE

\begin{abstract}
Basic analysis of the metric and extrinsic curvature tensors
\end{abstract}

\section{Purpose}

This thorn provides analysis routines to calculate the following quantities:

\begin{itemize}
\item
The trace of the extrinsic curvature ($trK$).
\item
The determinant of the 3-metric ($detg$).
\item
The components of the 3-metric in spherical coordinates \\
($g_{rr},g_{r\theta},g_{r\phi},g_{\theta\theta},g_{\phi\theta},g_{\phi\phi}$).
\item
The components of the extrinsic curvature in spherical coordinates \\
($K_{rr},K_{r\theta},K_{r\phi},K_{\theta\theta},K_{\theta\phi},K_{\phi\phi}$).
\item The components of the 3-Ricci tensor in cartesian coordinates \\
(${\cal R}_{ij}$) for $i,j \in \{1,2,3\}$.
\item The Ricci scalar (${\cal R}$).
\end{itemize}

\section{Trace of Extrinsic Curvature}

The trace of the extrinsic curvature at each point on the grid is placed in
the grid function {\tt trK}. The algorithm for calculating the trace 
uses the physical metric, that is it includes any conformal factor.

\begin{equation}
{\tt trK} \equiv tr K = \frac{1}{\psi^4} g^{ij} K_{ij}
\end{equation}

\section{Determinant of 3-Metric}

The determinant of the 3-metric at each point on the grid is placed in
the grid function {\tt detg}. This is always the determinant of the 
conformal metric, that is it does not include any conformal factor.

\begin{equation}
{\tt detg} \equiv det g =
-g_{13}^2*g_{22}+2*g_{12}*g_{13}*g_{23}-g_{11}*g_{23}^2-
g_{12}^2*g_{33}+g_{11}*g_{22}*g_{33}
\end{equation}


\section{Transformation to Spherical Cooordinates}

The values of the metric and/or extrinsic curvature in a spherical
polar coordinate system $(r,\theta,\phi)$ evaluated at each point on
the computational grid are placed in the grid functions ({\tt grr},
{\tt grt}, {\tt grp}, {\tt gtt}, {\tt gtp}, {\tt gpp}) and ({\tt krr},
{\tt krt}, {\tt krp}, {\tt ktt}, {\tt ktp}, {\tt kpp}).
In the spherical transformation, the $\theta$ coordinate is referred to
as {\bf q} and the $\phi$ as {\bf p}.

The general transformation from Cartesian to Spherical for such tensors 
is

\begin{eqnarray*}
A_{rr}&=&
\sin^2\theta\cos^2\phi A_{xx}
+\sin^2\theta\sin^2\phi A_{yy}
+\cos^2\theta A_{zz}
+2\sin^2\theta\cos\phi\sin\phi A_{xy}
\\
&&
+2\sin\theta\cos\theta\cos\phi A_{xz}
+2\sin\theta\cos\theta\sin\phi A_{yz}
\\
A_{r\theta}&=&
r(\sin\theta\cos\theta\cos^2\phi A_{xx}
+2*\sin\theta\cos\theta\sin\phi\cos\phi A_{xy}
+(\cos^2\theta-\sin^2\theta)\cos\phi A_{xz}
\\
&&
+\sin\theta\cos\theta\sin^2\phi A_{yy}
+(\cos^2\theta-\sin^2\theta)\sin\phi A_{yz}
-\sin\theta\cos\theta A_{zz})
\\
A_{r\phi}&=&
r\sin\theta(-\sin\theta\sin\phi\cos\phi A_{xx}
-\sin\theta(\sin^2\phi-\cos^2\phi)A_{xy}
-\cos\theta\sin\phi A_{xz}
\\
&&
+\sin\theta\sin\phi\cos\phi A_{yy}
+\cos\theta\cos\phi A_{yz})
\\
A_{\theta\theta}&=&
r^2(\cos^2\theta\cos^2\phi A_{xx}
+2\cos^2\theta\sin\phi\cos\phi A_{xy}
-2\sin\theta\cos\theta\cos\phi A_{xz}
+\cos^2\theta\sin^2\phi A_{yy}
\\
&&
-2\sin\theta\cos\theta\sin\phi A_{yz}
+\sin^2\theta A_{zz})
\\
A_{\theta\phi}&=&
r^2\sin\theta(-\cos\theta\sin\phi\cos\phi A_{xx}
-\cos\theta(\sin^2\phi-\cos^2\phi)A_{xy}
+\sin\theta \sin\phi A_{xz}
\\
&&
+\cos\theta\sin\phi\cos\phi A_{yy}
-\sin\theta\cos\phi A_{yz})
\\
A_{\phi\phi}&=&
r^2\sin^2\theta(\sin^2\phi A_{xx}
-2\sin\phi\cos\phi A_{xy}
+\cos^2\phi A_{yy})
\end{eqnarray*}

If the parameter {\tt normalize\_dtheta\_dphi} is set to {\tt yes}, 
the angular components are projected onto the vectors $(r d\theta, r \sin\theta d \phi)$ instead of the default vector $(d \theta, d\phi)$. That is,

\begin{eqnarray*}
A_{\theta\theta} & \rightarrow & A_{\theta\theta}/r^2
\\
A_{\phi\phi}& \rightarrow & A_{\phi\phi}/(r^2\sin^2\theta)
\\
A_{r\theta} & \rightarrow & A_{r\theta}/r
\\
A_{r\phi} & \rightarrow & A_{r\phi}/(r\sin\theta)
\\
A_{\theta\phi} & \rightarrow & A_{\theta\phi}/r^2\sin\theta)
\end{eqnarray*}

\section{Computing the Ricci tensor and scalar}
\label{sec:ricci}

The computation of the Ricci tensor uses the ADMMacros thorn. The
calculation of the Ricci scalar uses the generic trace routine in this
thorn.

% Do not delete next line
% END CACTUS THORNGUIDE

\end{document}
