\documentclass{article}

\begin{document}

\title{GRHydro\_Init\_Data}
\author{Luca Baiotti, Ian Hawke, Scott Hawley}
\date{24/07/2008}
\maketitle

% Do not delete next line
% START CACTUS THORNGUIDE

\abstract{GRHydro\_Init\_Data - some initial data for {\tt GRHydro}}

\section{Introduction}
\label{sec:intro}

This thorn generates some initial data for the GRHydro code. There are more (and more physically
interesting) initial-data codes in other thorns. As with the GRHydro code itself, please feel free to
add, alter or extend any part of this code. However please keep the documentation up to date (even,
or especially, if it's just to say what doesn't work).

Currently this thorn contains a few tests that should really be test suites, some shock-tube
initial data, 
%some (largely untested and unmaintained; please use the TOV initial-data codes in the
%dedicated thorns) TOV initial data solver routines, 
a routine to set atmopshere everywhere on the
grid (for tests), a routine to read initial data from files (not very generic, tough) and a routine
to set up the simple-wave initial data .


\subsection{Tests}
\label{sec:tests}

There are tests of the TVD reconstruction routine and of the routines
that convert between conservative and primitive variables. These all
just produce output to the screen or to {\tt fort.*} files. The
reconstruction test outputs the function to be reconstructed and the
boundary-extended values. The conservative-to-primitive test just
outputs the two sets of variables. If you haven't altered the code an if you set
\begin{verbatim}
eos_polytrope::eos_gamma =   2.0
eos_polytrope::eos_k     = 100.0
\end{verbatim}
(which are the defaults), the output should be
% I checked that in 2008 the number were still right (but one gets them with eos_gamma=2.0 and 
% eos_k ~ 0.1865) and committed the new ones for the default eos parameter values.

\begin{verbatim}
    primitive variables: 
    rho   :   1.29047172182043     
    velx  :   9.902578465178671E-004
    vely  :   9.902578465178671E-004
    velz  :   9.902578465178671E-004
    eps   :   0.374770481293314     
    press : 166.531726481819     
    w_lor :   1.00000147091915  
\end{verbatim}
The conservative to primitive to conservative test outputs the initial
and final data which should agree.

\subsection{Shocktube tests}
\label{sec:shock}

There are three possible shock-tube problems, referred to as {\tt Sod},
{\tt Simple} and {\tt Blast}, with initial data
\begin{center}
  \begin{tabular}[c]{|c|c|c|c|c|c|c|}
    \hline Type & $\rho_{_L}$ & $v^i_{_L}$ & $\varepsilon_{_L}$ & $\rho_{_R}$ & $v^i_{_R}$
    & $\varepsilon_{_R}$ \\ \hline
    Sod & 1 & 0 & 1.5 & 0.125 & 0 & 0.15 \\
    Simple & 10 & 0 & 20 & 1 & 0 & $10^{-6}$ \\
    Blast & 1 & 0 & 1500 & 1 & 0 & $1.5\cdot 10^{-2}$ \\ \hline
  \end{tabular}
\end{center}
The shock shape can be planar (along each axis or along the main diagonal) or spherical and the
position of the plane or of the center of the sphere can be chosen though parameters.
If a diagonal shock is selected, the initial data is set to either the left or right
state depending on where the centre of the cell falls. Cleverer
routines that weight the initial data to avoid ``staircasing'' may be
added if there is demand. For more discussion on shock tubes
see~\cite{livrevsrrfd}. 


%\subsection{TOV stars}
%\label{sec:tov}
%
%The Tolman-Oppenheimer-Volkoff solution is a spherically symmetric
%fluid ball matched to a Schwarzschild exterior. Typically an
%atmosphere is placed in the exterior to stop the equations of motion
%of the fluid being singular. Given an equation of state, the central
%density $\rho_c$ is specified. Then the solution is found by
%integrating the radial equations
%\begin{eqnarray}
%  \label{eq:tov}
%  \frac{\partial P(r)}{\partial r} & = & - \frac{(\rho + \rho \epsilon
%  + P)(m + 4\pi r^3 P)}{r (r - 2m)} \\
%  \frac{\partial (\log \alpha (r))}{\partial r} & = & \frac{m + 4 \pi
%  r^3 P}{r (r - 2m)} \\
%  \frac{\partial m(r)}{\partial r} & = & 4 \pi r^2 (\rho + \rho
%  \epsilon) \\ 
%  \gamma_{rr}(r) & = & \left( 1-\frac{2m(r)}{r} \right)^{-1},
%\end{eqnarray}
%where $m$ is the mass energy contained in a sphere radius $r$,
%$\gamma_{ij}$ the 3-metric, and $\alpha$ the lapse in standard
%Schwarzschild like coordinates. For more details see~\cite{hydro1}. 
%
%The routines here, written by Scott Hawley, use the LSODA library to
%integrate the equations and then interpolate onto the Cartesian
%grid. 
%
%This routine is untested and unmaintained: please use other thorns providing TOV initial data. They
%are located in dedicated thorns of this arrangement.

\subsection{Only atmosphere}
\label{sec:only-atmo}

For testing purposes, this routine sets all the points to the values of the atmosphere.


\subsection{Simple wave}
\label{sec:simple-wave}

This routine testes initial data for a simple wave with sinusoidal initial function for the velocity,
as described in Anile, Miller, Motta, {\it Formation and damping of relativistic strong
  shocks},Phys. Fluids {\bf 26}, 1450 (1983).



\begin{thebibliography}{1}

\bibitem{livrevsrrfd}
J.~M. Mart{\'{\i}} and E.~M{\"u}ller.
\newblock Numerical hydrodynamics in {S}pecial {R}elativity.
\newblock {\em Living Rev. Relativity}, {\bf 3}, 1999.
\newblock [Article in online journal], cited on 31/7/01,
  http://www.livingreviews.org/Articles/Volume2/1999-3marti/index.html.

\bibitem{hydro1}
J.~A. Font, M. Miller, W. Suen and M. Tobias.
\newblock Three Dimensional Numerical General Relativistic
Hydrodynamics I: Formulations, Methods, and Code Tests
\newblock {\em Phys. Rev.}, {\bf D61}, 044011, 2000.

\end{thebibliography}

% Do not delete next line
% END CACTUS THORNGUIDE

\end{document}


