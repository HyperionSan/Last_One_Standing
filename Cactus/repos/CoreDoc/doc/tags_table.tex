\section{Tags Tables: Optional Attributes for Cactus Grid Variables}

[This section last updated on $ $Date$ $.]

Cactus defines a ``tags table'' mechanism where a Cactus key-value
table is associated with each Cactus group of grid variables.
By default the flesh sets up an empty table for each group. If a group is
declared with a \verb|tags="..."| clause in its \verb|interface.ccl|,
the flesh uses this optional string to initialize the key-value tags table
for this group.
Keys in a tags table are conventionally taken to have case-insensitive
semantics.  The value string (which must be deliminated by a single or double
quotes in \verb|interface.ccl|)) is interpreted by the function
\verb|Util_TableSetFromString()| (see the Cactus Reference Manual
for a description of this function).

Currently the contents of a tags table are not evaluated by the CST
parser and not used by the flesh; it's entirely up to thorns to agree
among themselves as to what should be in the tags table and how it
should be interpreted.

You can get the tags table's table handle for a specified group
by calling the flesh function \verb|CCTK_GroupTagsTable()| or
\verb|CCTK_GroupTagsTableI()|.  Once you have this table handle,
you can access the tags table itself using the standard key-value
table API as documented in the Cactus Reference Manual.  

Most thorns which use tags-table information only look at the tags
table once (when they start up), and won't notice if the information
is changed later on.  Therefore, modifying the tags table is probably
not a good idea (it's likely to cause confusion if some thorns change
their behavior based on the new tags-table entries, while other thorns
don't change).

The following sections document some common conventions for tags-table
entries.

%%%%%%%%%%%%%%%%%%%%%%%%%%%%%%%%%%%%%%%%

\subsection{Tensor Types}

The following tags-table entries describe the tensor types of grid
functions/arrays.  These are widely used, both by \textbf{CactusEinstein}
and by other arrangements/thorns:
\begin{description}
\item[\texttt{tensortypealias}]\mbox{}\\
	This value associated with this tags-table key gives the
	primary tensor type (behavior under coordinate transformation),
	and may be one of the following strings:
	\begin{description}
	\item[\texttt{"scalar"}]
		for a 3-scalar or group of 3-scalars;
		this is generally assumed if no
		\texttt{tensortypealias} tag is present.
	\item[\texttt{"u"}]
		for a group of 3~grid functions/arrays
		storing a contravariant 3-vector $T^i$.
	\item[\texttt{"d"}]
		for a group of 3~grid functions/arrays
		storing a covariant 3-vector $T_i$.
	\item[\texttt{"uu\_sym"}]
		for a group of 6~grid functions/arrays
		storing a contravariant rank~2 symmetric 3-tensor $S^{ij}$.
	\item[\texttt{"dd\_sym"}]
		for a group of 6~grid functions/arrays
		storing a covariant rank~2 symmetric 3-tensor $S_{ij}$.
	\item[\texttt{"4scalar"}]
		for a 4-scalar or group of 4-scalars.
	\item[\texttt{"4u"}]
		for a group of 4~grid functions/arrays
		storing a contravariant 4-vector $T^a$.
	\item[\texttt{"4d"}]
		for a group of 4~grid functions/arrays
		storing a covariant 4-vector $T_a$.
	\item[\texttt{"4uu\_sym"}]
		for a group of 10~grid functions/arrays
		storing a contravariant rank~2 symmetric 4-tensor $S^{ab}$.
	\item[\texttt{"4dd\_sym"}]
		for a group of 10~grid functions/arrays
		storing a covariant rank~2 symmetric 4-tensor $S_{ab}$.
	\end{description}
	Note that for multi-index tensors, different thorns have different
	conventions about the order of the grid functions.  Row-major
	order by indices is most common (for example, indices 11, 12,
	13, 22, 23, 33 for a rank~2 symmetric 3-tensor), but some thorns
	may use column-major order instead (in this example, indices
	11, 21, 31, 22, 23, 33).  This also illustrates that for multi-index
	symmetric tensors, there are multiple conventions about which
	subset of algebraically-independent components is actually stored.
\item[\texttt{tensormetric}]\mbox{}\\
	The string value associated with this tags-table key gives the
	full name (Thorn::Gridfn) of the grid variable holding the
	tensor 3-metric with respect to which the other tensor-transformation
	properties are defined.
\item[\texttt{tensorweight}]\mbox{}\\
	For a tensor density, this gives the weight.
	If omitted, most code will use a default weight of $0$.
\item[\texttt{tensorparity}]\mbox{}\\
	For a (pseudo)tensor, this gives the parity.
        The parity can be either $+1$ or $-1$.
        The parity defines whether the quantity has a different sign
        change behaviour under odd parity transformations, such as
        e.g.\ reflections.  E.g., Pseudoscalars and axial vectors have
        negative parity.
	If omitted, most code will use a default parity of $+1$.
\item[\texttt{tensorspecial}]\mbox{}\\
	This is an ``escape-hatch'' tag used for things which are
	``sort of tensors, but not really''.  If present, it may
	have one of the following string values describing variables
	in the AEI BSSN formalism (gr-qc/0003071):
	\begin{description}
	\item[\texttt{"phi"}]
		for the BSSN logarithmic-conformal-factor
		$\phi \equiv \frac{1}{12} \log \det [g_{ij}]$.
	\item[\texttt{"gamma"}]
		for the BSSN contracted-conformal-Christoffel-symbols
		$\tilde{\Gamma}^i \equiv \tilde{g}^{mn} \tilde{\Gamma}^i{}_{mn}
				  \equiv -\partial_j \tilde{g}^{ij}$.
	\end{description}
\end{description}

Some thorns, particularly those for multi-patch computations, may also
use the following tags-table entry:
\begin{description}
\item[\texttt{tensorbasis}]\mbox{}\\
	The string value associated with this tags-table key specifies
	the type of tensor basis being used:
	\begin{description}
	\item[\texttt{"global xyz"}]
		means that the tensor basis is a global Cartesian one,
		the same for all patches (and hence that this group
		should not be tensor-transformed when it's interpolated
		from one patch to another)
	\item[\texttt{"local"}]
		means that the tensor basis is a local per-patch one
		(which varies from one patch to another); usually this
		will be the obvious local-Cactus-coordinates coordinate basis
	\end{description}
	Other values may also be allowed for this key; see the documentation
	for individual multi-patch infrastructures for further details
	(including a discussion of what defaults are assumed for this).
\end{description}

%%%%%%%%%%%%%%%%%%%%

\subsubsection{Example}

For example, the BSSN 3-metric might be declared with an \verb|interface.ccl|
entry like this:
\begin{verbatim}
real conformal_metric                           type=GF timelevels=3    \
     tags='tensortypealias="dd_sym"                                     \
           tensorweight=-0.66666666666666667                            \
           tensormetric="BSSNBase::conformal_metric"'
{
g_tilde_dd_11
g_tilde_dd_12
g_tilde_dd_13
g_tilde_dd_22
g_tilde_dd_23
g_tilde_dd_33
} "conformal 3-metric $\tilde{g}_{ij}$"
\end{verbatim}
Notice the syntax here: the \verb|tags=| string is enclosed in single
quotes (\verb|tags='...'|), and may contain double-quoted strings.

%%%%%%%%%%%%%%%%%%%%%%%%%%%%%%%%%%%%%%%%

\subsection{Checkpointing of Cactus Grid Variables}

During checkpointing all variables which have global storage assigned are saved
in a checkpoint file for later recovery. For some variables this isn't really
necessary because they are set up at \verb|BASEGRID| and remain constant
thereafter, or they are used only as temporaries which don't need to be
initialized from a checkpoint during recovery.

\begin{description}
\item[\texttt{checkpoint}]\mbox{}\\
  This boolean key-value tag is meant as a hint to checkpoint methods
  as to whether a Cactus grid variable group needs to be checkpointed.
  \begin{description}
    \item[\texttt{"yes"}]\mbox{}\\
      The group needs to be saved in a checkpoint.
      This is the default if no such tag is specified in a group's tags table.
    \item[\texttt{"no"}]\mbox{}\\
      The group may be omitted from a checkpoint. For a recovery Cactus run,
      user code in application thorns makes sure that variables in this group
      are properly initialized.
  \end{description}
\end{description}

%%%%%%%%%%%%%%%%%%%%%%%%%%%%%%%%%%%%%%%%

\subsection{Carpet-specific Tags}

This section lists tags table entries which are specific to the mesh-refinement
driver thorn {\tt Carpet}. For a detailed description of these tags please
refer to the Carpet documentation.

\begin{description}
\item[\texttt{Prolongation}]\mbox{}\\
  This string key-value tag speficifies which method Carpet should use to
  prolongate variables in this group. Possible values are:
  \begin{description}
    \item[\texttt{"none"}]\mbox{}\\
      do not prolongate this group
    \item[\texttt{"Copy"}]\mbox{}\\
      use simple copying for prolongation (needs only one time level)
    \item[\texttt{"Lagrange"}]\mbox{}\\
      use Lagrange interpolation (this is the default if no prolongation
      operator has been specified for this group)
    \item[\texttt{"TVD"}]\mbox{}\\
      use TVD stencils (for hydro)
    \item[\texttt{"ENO"}]\mbox{}\\
      use ENO stencils (for hydro)
    \item[\texttt{"WENO"}]\mbox{}\\
      use WENO stencils (for hydro)
  \end{description}

\item[\texttt{ProlongationParameter = <parameter\_name>}]\mbox{}\\
  Rather than naming a fixed prolongation operator at compile time via the
  {\tt Prolongation} tag in an interface.ccl file, it is also possible to
  specify it for each group only at runtime: the {\tt ProlongationParameter}
  tag key expects a string value denoting the full name of a Cactus {\bf STRING}
  or {\bf KEYWORD} parameter which specifies the actual prolongation operator
  (from the same set listed above).\\
  The {\tt ProlongationParameter} and {\tt Prolongation} tags are mutually
  exclusive, one can either use one or the other in an inteface.ccl file.\\
  Currently Carpet evaluates the prolongation parameter information only once
  at startup, it is assumed that prolongation operators for individual groups
  do not change afterwards.

%\item[\texttt{Interpolator = <name>}]\mbox{}\\
%  This tag speficies ???

\item[\texttt{InterpNumTimelevels = <levels>}]\mbox{}\\
  This tag specifies how many timelevels should be used for time interpolation.

\end{description}


%%%%%%%%%%%%%%%%%%%%%%%%%%%%%%%%%%%%%%%%
